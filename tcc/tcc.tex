\documentclass{iiufrgs}
\usepackage[utf8]{inputenc}
\usepackage{graphicx}
\usepackage[section]{placeins}
\usepackage{setspace}
\usepackage{fontenc}
\usepackage{listings}
\usepackage{color}
\usepackage{url}
\usepackage[printonlyused]{acronym}
\usepackage{rotating}
\usepackage{bytefield}
\usepackage[table]{xcolor}
\usepackage{multirow}
\usepackage{subfigure}
\usepackage{lscape}
\usepackage{enumitem}
\usepackage{fixltx2e}
\usepackage{longtable}
\usepackage{mathtools}
\usepackage{tabularx}
\usepackage{adjustbox}
\usepackage{amsmath}
\usepackage{booktabs}
\usepackage{textcomp}
\usepackage{array}
\usepackage{booktabs}
\usepackage{xcolor}
\usepackage{colortbl}
%\usepackage[brazilian]{babel}
\DeclarePairedDelimiter{\abs}{\lvert}{\rvert}
\onehalfspacing
\urlstyle{sf}
\setdescription{topsep=1em,parsep=0pt,partopsep=0pt,itemsep=0pt}
\setitemize{topsep=1em,parsep=0pt,partopsep=0pt,itemsep=0pt}
\setenumerate{topsep=1em,parsep=0pt,partopsep=0pt,itemsep=0pt}
\lstset{frame=tb,
  language=Java,
  aboveskip=3mm,
  belowskip=3mm,
  showstringspaces=false,
  columns=flexible,
  basicstyle={\small\ttfamily},
  numbers=none,
  numberstyle=\tiny\color{gray},
  keywordstyle=\color{blue},
  commentstyle=\color{dkgreen},
  stringstyle=\color{red},
  breaklines=true,
  breakatwhitespace=true,
  tabsize=3
}

\course{\cgcc}
\title{O Uso de Processamento de Linguagem Natural para a Análise de
Sentimentos na Rede Social Reddit.}
\author{Santos Andreata}{Guilherme Henrique}
\advisor[]{Martinotto}{André Luis}
\location{Caxias do Sul}{}
\bibpunct{(}{)}{;}{a}{,}{,}

\begin{document}

\maketitle  

\begin{titlepage}
%\setcounter{page}{2} - Inclui o número da página
%\thispagestyle{headings}
\vfill

\begin{center}
{\setlength{\unitlength}{1cm}\makebox(12,6.5){\parbox[c]{12cm}{\setlength{\parskip}{0.8cm}\center\vskip -1.2cm\LARGE{\bf O Uso de Processamento de Linguagem Natural para a Análise de Sentimentos na
Rede Social Reddit.}\par \normalsize por\par \large Guilherme Henrique
Santos Andreata\par}}}
\end{center}

{\large Projeto de Diplomação submetido ao curso de Bacharelado em Sistemas de
Informação da área de conhecimento de ciências exatas e engenharia, como requisito obrigatório para graduação.}

\vfill

\begin{center}
{\Large\bf Projeto de Diplomação}
\end{center}

\vfill

\begin{singlespace}
Orientador: {André Luis Martinotto\par}

Banca examinadora:\par
\hspace{1cm} {\setlength{\unitlength}{1cm}
\makebox(9,1){\parbox[c]{9cm}{\center Daniel Luis Notari\\ CCTI/UCS}}}\par
\hspace{1cm} {\setlength{\unitlength}{1cm}
\makebox(9,1){\parbox[c]{9cm}{\center Helena Graziottin Ribeiro\\ CCTI/UCS}}}\par

\vfill

%\hfill{\setlength{\unitlength}{1cm}\makebox(9,2.5){\parbox[c]{9cm}{\setlength{\parskip}{0.8cm}\center\vskip
% -1.2cm Projeto de Diplomação apresentado em\\ 5 de Dezembro de 2013\par Daniel Luís Notari\\ Coordenador}}}

\end{singlespace}

\end{titlepage}

\tableofcontents

\chapter*{Lista de acrônimos}

\vspace{20px}
\begin{acronym}[XXXXXXXXXX]
\acro{NLP}[NLP]{\textit{Natural Language Processing}}
\acro{NLTK}[NLTK]{\textit{Natural Language Toolkit}}
\acro{MaxEnt}[MaxEnt]{\textit{Maximum Entropy}}
\acro{RNTN}[RNTN]{\textit{Recursive Neural Tensor Networks}}
\acro{VADER}[VADER]{\textit{Valence Aware Dictionary and sEntiment Reasoner}}
\acro{JSON}[JSON]{\textit{JavaScript Object Notation}}
\acro{POJO}[POJO]{\textit{Plain Old Java Objects}}
\acro{SVM}[SVM]{\textit{Support Vector Machines}}
\end{acronym}
\listoffigures
\listoftables

\keyword{Reddit}
\keyword{Processamento de Linguagem Natural}
\keyword{Análise de Sentimentos}

\begin{abstract}

Nos tempos atuais, a sociedade tem cada vez mais se expressado através de Redes
Sociais, dentre várias redes sociais, se destaca o Reddit, no qual além de ser
uma das maiores redes sociais no mundo, é uma rede social de \textit{links},
aonde usuários postam e comentam sobre estes gerando um grande volume de dados que muitas vezes se dão por
ignorados.

A identificação de padrões de sentimentos expressos por grupos dessa comunidade, se faz útil visto que a partir dessa avaliação é possível construir ferramentas que apoiam decisões tanto de um ponto de vista político, como por
exemplo, entender qual é a opinião sobre um determinado assunto de um conjunto
de eleitores, tanto quanto um ponto de vista de negócios, para entender qual a
opinião dos consumidores de um produto, ou de seu competidor, a respeito de um
determinado assunto.

Neste volume serão apresentados estudos sobre o Processamento de Linguagem
Natural e Análise de Sentimentos, os quais apresenta o objetivo da criação de um
\textit{software} no qual seja possível efetuar a análise de sentimentos na rede
social Reddit.

\end{abstract}


\acresetall

\include{capitulos/introducao}
\chapter{Processamento de Linguagem Natural}
\label{cap:Processamento}

% Um computador, obviamente, está preparado para entender sua própria linguagem,
% como por exemplo, um compilador interpreta linhas de código fonte para gerar um
% programa executável seguindo exatamente o algoritmo utilizado. Por isso, temos o
% termo Natural no Processamento de Linguagem.

O objetivo da área de Processamento de Linguagem Natural é analisar a linguagem
natural, ou seja, a linguagem utilizada pelo seres humanos seja ela escrita
ou falada \cite{manningschutze1999}.

O Processamento de Linguagem Natural é uma área antiga, sendo anterior a
invenção dos computadores modernos. De fato, sua primeira grande aplicação foi
um dicionário desenvolvido no Birkbeck College em Londres no ano de 1948. Por ser
uma área complexa, seus primeiros trabalhos foram notavelmente falhos o que
causou uma certa hostilidade por parte das agências fomentadoras de pesquisas.

Os primeiros pesquisadores eram muitas vezes bilíngues, como por exemplo,
nativos alemães que imigraram para os Estados Unidos. Acreditava-se que pelo
fato desses terem conhecimento de ambas as linguas, Ingles e Alemão, eles teriam
capacidade de desenvolver programas de computadores que efetuariam a tradução
de modo satisfatório. Uma vez que esses encontraram muitas dificuldades,
ficou claro que o maior problema não era o conhecimento das
línguas, e sim como expressar esse conhecimento na forma de um programa de
computador \cite{history}.

Para que um computador seja capaz de interpretar uma
língua, primariamente necessitamos compreender como nós efetuamos essa
interpretação.
Por isso, uma parte considerável do Processamento de Linguagem Natural está apoiado na área de Linguística.

\section{Linguística}

O objetivo da Linguística é compreender como os seres humanos adquirem, produzem
e entendem as diversas línguas, ou seja, a forma como conversamos, a nossa
escrita e outras mídias de comunicação \cite{manningschutze1999}.

Na linguagem tanto escrita, como na falada, existem regras que são utilizadas
para estruturar as expressões. Uma série de dificuldades no Processamento de
Linguagem Natural são ocasionadas pelo fato de que as pessoas constantemente
mudam essas regras para satisfazerem suas necessidades de comunicação
\cite{manningschutze1999}. Uma vez que as regras são constantemente modificadas
pelo locutor, se torna extremamente difícil a criação de um software ou hardware
que efetue a interpretação de uma língua.


% \subsection{Sintaxe e Semântica}
%
% No seu livro Estruturas Sintáticas, Noam Chomsky cita as seguintes frases
% ``Ideias verdes incolores dormem furiosamente'' e ``Incolores verde ideias dormem
% furiosamente''.
%
% A primeira frase, do ponto de vista sintático é correta, porém, assim como a
% segunda frase, semânticamente não faz sentido.
%
% O fato de que podemos modificar as regras da lingua de duas formas distintas é
% utilizado como evidência para a separação da sintaxe e semântica na língua.
% \cite{jacksonmoulinier2007}

\section{Métodos de Processamento de Linguagem Natural}

O \ac{NLP} tem como objetivo a execução de diferentes tarefas, como por exemplo,
a categorização de documentos, a tradução e a geração de textos a partir de um
banco de dados, etc. Podemos citar duas classes de métodos para a execução deste
tipo de tarefas, que são os métodos simbólicos e os métodos estatísticos.

Nos final dos anos 50 e 60, existiam excelentes métodos estatísticos, que foram
desenvolvidos durante a segunda guerra mundial, para a solução de problemas
Linguísticos \cite{shannon48}.
Porém, no ano de 1957, Chomsky publicou o trabalho intitulado de
\textit{``Syntactic Structures''} onde descreve a
teoria da gramática gerativa, que é uma teoria que considera a
gramática como um conjunto de regras. Essa abordagem através de um conjunto de
regras, ao invés de um modelo matemático, entra em conflito com os trabalhos
anteriores, criando duas comunidades no campo da Linguística. Como reflexo
dessas duas comunidades, a área de \ac{NLP} que crescia em paralelo, também foi
dividida em duas áreas. A primeira dessas áreas que fazia uso de métodos
baseados em regras (simbólica) e a segunda que fazia o uso de métodos quantitativos (estatísticas).


Nesta seção será apresentado um exemplo de método simbólico e de um método
estatístico.
Destaca-se que essa descrição apresenta como objetivo, apenas
diferenciar ambas as classes de métodos, através de seus requisitos e forma de execução.
Destaca-se ainda que os métodos apresentados nesta seção não são utilizados na
análise de sentimentos, sendo que os métodos específicos para essa
identificação serão descritos no Capítulo \ref{cap:Classificadores}.


\subsection{Método Simbólico}
O método simbólico ou racionalista está
baseado no campo da Linguística e faz o uso da manipulação dos símbolos,
significados e das regras de um texto. Um exemplo simples de um método simbólico
é o método de Brill \cite{Brill:1992:SRP:974499.974526}. Por exemplo, no método de
Brill a frase ``João pintou a casa de branco'', será separada em palavras que
serão classificadas através de um dicionário pré-definido, como:

\begin{table}[htb]
\centering
\begin{tabular}{l|l|l|l|l|l|l}
Palavra         & João        & pintou & a      & casa        & de
& branco
\\
%Correta: & Substantivo & Verbo  & Artigo & Substantivo & Preposição &
% Substantivo \\
Classificação:   & 			   & Verbo  & Artigo & Substantivo & Preposição & Adjetivo
\end{tabular}
\label{my-label}
\end{table}

Observa-se que algumas palavras não foram
identificadas, como ``João'', ou classificadas de forma incorreta, como
``branco". Desta forma, o método de Brill utiliza-se de outras duas regras para
a classificação.
A primeira regra classifica todas as palavras desconhecidas que iniciam com uma
letra em maiúscula como substantivos, por exemplo, a palavra ``João''. Já a
segunda regra, atribui para a palavra desconhecida a mesma classificação de outras palavras que terminam com as mesmas três letras. Por exemplo, supondo
que a palavra ``pintou'' não fosse encontrada no dicionário, essa seria
associada a outras palavras terminadas com o sufixo ``tou'', ou seja, essa seria
classificada como verbo.

\begin{table}[htb]
\centering
\begin{tabular}{l|l|l|l|l|l|l}
Palavra         & João        & pintou & a      & casa        & de
& branco
\\
%Correta: & Substantivo & Verbo  & Artigo & Substantivo & Preposição &
% Substantivo \\
Classificação:   & \textbf{Substantivo} & Verbo  & Artigo & Substantivo &
Preposição & Adjetivo
\end{tabular}
\label{my-label}
\end{table}



Após essa classificação inicial, o método executa o seguinte conjunto de
regras, ou ainda, regras derivadas dessas:

\begin{itemize}
  \item Se uma palavra tem a classificação \textbf{A} e está no contexto
  \textbf{C} então a sua classificação deverá ser mudada para \textbf{B}. Por
  exemplo, se uma palavra \textbf{A} (branco no exemplo) é um adjetivo e uma das
  duas palavras anteriores é uma preposição (``de'' no contexto \textbf{C}
  ), mude a sua classificação para um substantivo (classificação \textbf{B}).
  
  \[\overbrace{\text{João}}^\text{Substantivo}
  \overbrace{\text{pintou}}^\text{Verbo}
  \overbrace{\text{a}}^\text{Artigo}
  \underbrace{
  \overbrace{\text{casa}}^\text{Substantivo}
  \overbrace{\text{de}}^\text{Preposição}}_\text{Contexto \textbf{C}}
  \underbrace{\overbrace{\text{branco}}^{\textcolor{red}{Adjetivo}}}_\text{Classificação
  \textbf{A}\textrightarrow\textbf{B}}
  \]
  
  \item Se uma palavra tem a classificação \textbf{A} e tem uma propriedade
  \textbf{P} então a sua classificação deverá ser alterada para \textbf{B}. Por
  exemplo, se uma palavra \textbf{A} (``Linda'') foi classificada como um
  adjetivo e é iniciada com uma letra maiúscula (propriedade \textbf{P}), sua
  classificação deverá ser alterada para substantivo (classificação \textbf{B}).
  
  \[\overbrace{\text{Comprei}}^\text{Verbo}
  \overbrace{\text{flores}}^\text{Substantivo}
  \overbrace{\text{para}}^\text{Preposição}
  \underbrace{\overbrace{\text{L}\text{inda}}^{\textcolor{red}{Adjetivo}}}_\text{Classificação
  \textbf{A}\textrightarrow\textbf{B}}
  \]
  
  \item Se uma palavra tem a classificação \textbf{A} e uma palavra com a
  propriedade \textbf{P} está na região \textbf{R}, sua classificação deverá
  ser \textbf{B}. Por exemplo, se uma das duas palavras anteriores à palavra
  ``Linda'' (``João adora" na região \textbf{R}) iniciam com letra maiúscula
  (propriedade \textbf{P}), sua classificação deverá ser alterada para substantivo (classificação \textbf{B}).
  
  \[\underbrace{\overbrace{\text{João}}^\text{Substantivo}
  \overbrace{\text{adora}}^\text{Verbo}}_\text{Região \textbf{R}}
  \underbrace{\overbrace{\text{L}\text{inda}}^{\textcolor{red}{Adjetivo}}}_\text{Classificação
  \textbf{A}\textrightarrow\textbf{B}}
  \]
  
  
\end{itemize}

\subsection{Método Estatístico}
Um método estatístico utiliza-se de uma grande
quantidade de texto, procurando por padrões e
associações a modelos, sendo que esses padrões podem ou não estar relacionados
com regras sintáticas ou semânticas.

Os métodos estatísticos baseia-se na utilização de um sistema de aprendizado
supervisionado, ou seja, a classificação é feita a partir de um conjunto de dados já
classificado, que é chamado de \textit{training set}. Um exemplo de método
estatístico é a utilização de Modelos de Markov com a aplicação do algoritmo de
Viterbi \cite{manningschutze1999}.

Em um Modelo de Markov, a classificação da frase ``João comprou um
carro'' é feita a partir de um \textit{training set} que pode, por exemplo, ser
composto por textos retirados de \textit{web-sites}, sendo que as palavras
destes textos já devem estar classificadas. A partir deste \textit{training
set}, as palavras ``João'', ``comprou'' e ``carro'' seriam classificadas como
substantivo, verbo e substantivo, respectivamente. Já a palavra ``um'' apresenta uma ambiguidade uma vez que pode
ser classificada como um artigo (ART), ou um substantivo (SM) ou um pronome
(PRO).
A Figura \ref{fig:markov} ilustra o conjunto de possibilidades criadas pelo
classificador para a classificação completa da frase.

\begin{figure}[htbp]
\centering
\includegraphics[height=180px]{imagens/markov.png}
\caption{Caminhos possíveis de classificação}
\label{fig:markov}
\end{figure}

A idéia central da utilização de Modelos de Markov é
escolher, entre os caminhos possíveis (Figura \ref{fig:markov}), o caminho
de maior probabilidade. Para tanto, se faz necessário calcular a probabilidade de todos
os caminhos através de um Modelo de Markov. Após, utiliza-se o
Algoritmo de Viterbi para definir qual o caminho com maior probabilidade
\cite{manningschutze1999}.

O Modelo de Markov irá utilizar-se do \textit{training set} para inferir a
classificação da palavra ``um''. Por exemplo, considerando-se um
\textit{training set} hipotético com as seguintes características: 10000
substantivos aonde 150 são a palavra ``um''; 10
são a palavra ``João''; 50 são a palavra ``carro''; 20000 artigos aonde 500 são
a palavra ``um''; 12000 verbos aonde 50 são a palavra ``comprou''; 15000
pronomes aonde 50 são a palavra ``um''. Neste caso, a probabilidade da palavra
ser um substantivo é dada pela Equação \ref{eq:associacao}, onde no
\textit{training set} temos 150 instâncias da palavra ``um'' classificadas como
substantivo e um total de 10000 substantivos.
\begin{equation}
\begin{split}
P(palavra|classe) = \frac{C(classe,palavra)}{C(classe)}  \\
P(um|SM) = \frac{C(SM,um)}{C(SM)} = \frac{150}{10000} = 0,015.
\end{split}
\label{eq:associacao}
\end{equation}

Desta forma tem-se que a probabilidade de ``um'' ser um substantivo é de 0,015.
A Equação \ref{eq:associacao} também é aplicada para as demais classes,
neste caso, pronome ou artigo. Por exemplo, a probabilidade da palavra ``um'' ser um pronome
seria 0,0033 e a probabilidade da palavra ``um'' ser um artigo seria 0,025. Esse
cálculo de probabilidade é realizado para todas as palavras da frase que está
sendo classificada. Na Tabela \ref{tabela:associacao} tem-se os resultados
obtidos para todas as palavras da frame ``João comprou um carro''.

\begin{table}[htb]
\centering
\begin{tabular}{|l|l|l|l|l|}
\hline
& João  & comprou & um     & carro  \\ \hline
Substantivo & 0.001 & 0       & 0.015  & 0.005  \\ \hline
Verbo       & 0     & 0.0042  & 0      & 0      \\ \hline
Artigo      & 0     & 0       & 0.025  & 0      \\ \hline
Pronome     & 0     & 0       & 0.0033 & 0      \\ \hline
\end{tabular}
\caption{Tabela de Probabilidades de Associação}
\label{tabela:associacao}
\end{table}

Além da probabilidade de associação a uma determinada classe, é calculada a
probabilidade de transição de uma classe para a outra. Neste caso, o
\textit{training set} hipotético apresenta as seguintes características:

\begin{itemize}
  \item De 20000 frases, 2500 iniciam com um substantivo, 5000 iniciam com um
  verbo, 5000 iniciam com um artigo e 5000 iniciam com um pronome.
  \item De 10000 substantivos, 10000
  são seguidos por verbos.
  \item De 12000 verbos, 3000 são seguidos por um substantivo, 2000
  são seguidos por um outro verbo, 5000 são seguidos por um artigo e 2000 são
  seguidos por um pronome.
  \item De 20000 artigos, 20000 são seguidos por um substantivo.
  \item De 15000 pronomes, 10000 são seguidos por um substantivo e 5000 são
  seguidos por um verbo.
  
  
\end{itemize}

Neste caso, a probabilidade de transição de um verbo para um substantivo é dada
pela Equação \ref{eq:transicao}, onde no \textit{training set} existem 12000
verbos, os quais 3000 são seguidos por um substantivo, desta forma para a
transição de um verbo para substantivo tem-se:

\begin{equation}
\begin{split}
P(transicao|classe) = \frac{C(classe,transicao)}{C(classe)} \\
P(SM|VB) = \frac{C(VB,SM)}{C(VB)} = \frac{3000}{12000} = 0,25
\end{split}
\label{eq:transicao}
\end{equation}

Da mesma forma, a probabilidade de transição é cálculada para todas as
demais classes. Por exemplo, a probabilidade de
transição de um verbo para outro verbo é 0,17, de um verbo para um artigo é 0,42
e de um verbo para um pronome é 0,17. Também, a Equação \ref{eq:transicao} é
utilizada para o cálculo da probabilidade da frase iníciar com determinada
classe.
A Tabela \ref{tabela:transicao} tem-se a probabilidade de transição para todas
as classes do \textit{training set} de exemplo.

\begin{table}[htb]
\centering
\begin{tabular}{|l|l|l|l|l|}
\hline
& Substantivo & Verbo & Artigo & Pronome \\ \hline
Início      & 0.125       & 0.25  & 0.25   & 0.25    \\ \hline
Substantivo & 0.0         & 1.0   & 0.0    & 0.0     \\ \hline
Verbo       & 0.25        & 0.17  & 0.42   & 0.17    \\ \hline
Artigo      & 1.0         & 0.0   & 0.0    & 0.0     \\ \hline
Pronome     & 0.67        & 0.33  & 0.0    & 0.0     \\ \hline
\end{tabular}
\caption{Tabela de Probabilidade de Transição}
\label{tabela:transicao}
\end{table}

A partir das probabilidades calculadas através do Modelo de Markov, é
utilizado o algoritmo de Viterbi para determinar o caminho mais provável.

\begin{equation}
\begin{split}
v_t(j) = v_{t-1} a_{ij} b_j(o_t)
\end{split}
\label{eq:viterbi}
\end{equation}

O caminho mais provável é obtido através da Equação \ref{eq:viterbi}, sendo que
essa é aplicada a todas as palavras da frase. Na Equação \ref{eq:viterbi} os
termos $v_t$, $v_{t-1}$, $a_{ij}$ e $b_j(o_t)$ correspondem respectivamente a
probabilidade do caminho atual, resultado do caminho anterior, a probabilidade
de transição e a probabilidade de associação.
Portanto a palavra ``João'', $v_{t-1}$ é representada pelo valor 1, visto
que essa é a primeira palavra, ou seja, não foram calculados os valores de
$v_t$ para as palavras anteriores, $a_{ij}$ é a probabilidade de transição entre
``Início'' e um substantivo (Tabela \ref{tabela:transicao}) e $b_j(o_t)$ é a
probabilidade de associação da palavra João com substantivo (Tabela
\ref{tabela:associacao}). Desta forma tem-se que $v_t$ para a palavra João é:

\begin{equation}
\begin{split}
v_t(j) = 1 * 0,125 * 0,001 = 0,000125.
\end{split}
\label{eq:joao}
\end{equation}

Já para a palavra ``comprou'' tem-se:
\begin{equation}
\begin{split}
v_t(j) = 0,000125 * 1 * 0,0042 = 0,000000525.
\end{split}
\label{eq:comprou}
\end{equation}

Aonde além das probabilidades de transição e associação respecitvamente
retirados das Tabelas \ref{tabela:transicao} e \ref{tabela:associacao}, $v_{t-1}$ é representado pelo cálculo do
caminho anterior, ou seja, 0,000125. Ao efetuar o cálculo de todos os caminhos,
para determinar qual a classificação correta de uma palavra, é escolhido o caminho que tem maior probabilidade, no
caso apresentado, a palavra ``um'' é classificada como artigo.

\begin{figure}[htbp]
\centering
\includegraphics[height=180px]{imagens/markov2.png}
\caption{Caminhos já decididos de classificação}
\label{fig:markov2}
\end{figure}

Como visto, o método simbólico para resolver problemas de Processamento de
Linguagem Natural faz uso da criação de regras baseadas no conhecimento humano,
enquanto o método estatístico, decide através de cálculos probabilísticos
apoiados em estatísticas de um banco de dados para a resolução correta do
problema.


% Uma maneira de diferenciarmos os dois métodos é através do problema de
% ambiguidade. Por exemplo, nas frases:
%
% ``João entrou no carro conversível de óculos novos.''. E ``João entrou no carro
% conversível de farol apagado.''.
%
% Em ambas as frases, após a preposição ``de'' segue um substantivo masculino.
% Porém, cada uma das frases se refere a um substantivo diferente. A
% primeira se refere ao João, visto que não existe sentido em um carro ter óculos.
% Já a segunda se refere ao próprio carro, visto que não existe sentido em João
% ter faróis.
%
% O método simbólico para resolver esse problema faz a criação de novas regras se
% baseadas no conhecimento humano para a solução de qual o significado da frase.
% Já o método estatístico, irá verificar qual a probabilidade de cada significado
% para cada frase através de análises similares decidindo através de metódos
% estatísticos qual o significado correto para cada frase
% \cite{jacksonmoulinier2007}.

\chapter{Métodos estatísticos e simbólicos aplicados na análise de sentimentos}
\label{cap:Classificadores}

Antigamente, para sabermos a opinião de outras pessoas sobre um
determinado produto, tinhamos que perguntar diretamente. Com a
popularização da Internet e também de redes sociais, milhares de pessoas
compartilham para todos as suas opinões sobre produtos, política, serviços e
demais itens sujeitos a nossa crítica. Porém, muitas vezes essas opiniões acabam
por ser esquecidas pela dificuldade de se analisar uma grande quantidade de
textos. Como saber a opinião geral sobre determinado produto em uma seção de
comentários com mais de 1000 opiniões diferentes? A análise de sentimentos,
considerada uma tarefa do \ac{NLP}, tem como função identificar e quantificar
esses sentimentos expressos através de textos.

Outros possíveis métodos de análise de sentimentos através de aprendizado de
máquina são \ac{SVM} e \ac{MaxEnt}, os quais possuem
performance similar ao Naive Bayes \cite{domingos97naivebayes}.


Neste capítulo serão descritos um método estatístico, Naive Bayes e um método
simbólico, \ac{VADER}, aplicados na análise de sentimentos. Outros possíveis
métodos estatísticos para a análise de sentimentos são \ac{SVM} e \ac{MaxEnt},
os quais possuem performance similar ao Naive Bayes \cite{Pang:2002:TUS:1118693.1118704}. O método simbólico
\ac{VADER} foi selecionado pois este, segundo seus criadores, apresenta
performance superior aos métodos já existentes \cite{conf/icwsm/HuttoG14}.

Ambos métodos selecionados consideram o uso da língua Inglesa,
visto que o \textit{Website} analisado (Reddit) possui a maioria de seus
comentários nessa língua e não foram encontrados métodos que façam uso da língua
Portuguesa com similar precisão.




%Para o \ac{NLP} e também para o campo de estatísticas, classificadores são
%algorítmos que identificam a qual categoria determinado item pertence. Essa
%classificação é feita a partir de dados já classificados corretamente, ou seja,
%um \textit{training set}.

\section{Naive Bayes}

O Naive Bayes é um método estatístico para a classificação o qual podemos
aplicar para a análise de sentimento. Ele faz uso do teorema de Bayes para
que a partir de um \textit{training set} se possa inferir uma classificação.

Por exemplo, precisamos determinar se a seguinte frase demonstra um sentimento
negativo ou positivo: ``This place is great.'', como este método faz uso de um
\textit{training set} para a classificação, será considerado o seguinte \textit{training set} hipotético:

\begin{table}[htb]
\centering
\begin{tabular}{|l|l|}
\hline
Texto  & Categoria \\ \hline
The food was great  & Positiva     \\ \hline
They are horrible!    & Negativa     \\ \hline
I love the food here  & Positiva     \\ \hline
This place is wonderful  & Positiva     \\ \hline
Forgettable experience  & Negativa     \\ \hline
\end{tabular}
\caption{\textit{Training Set}}
\label{tab:trainingsetnb}
\end{table}

A partir do \textit{training set} representado pela tabela
\ref{tab:trainingsetnb} o método irá calcular a probabilidade da frase ``This
place is great'' ser positiva e também a possibilidade dela ser uma frase
negativa e a partir dessas duas possibilidades, será escolhida a maior.


\subsection{Teorema de Bayes}

Para cálcular a probabilidade da frase ``This place is great'' pertencer a cada categoria
utilizando o teorema da Bayes, é aplicada a Equação \ref{eq:teorema}:
\begin{equation}
\begin{gathered}
P(c|d) = \frac{P(d|c) \times P(c)}{P(d)} \\
\label{eq:teorema}
\end{gathered}
\end{equation}

Os termos da Equação \ref{eq:teorema} são determinados da seguinte forma:
\begin{itemize}
  \item P(c$\vert$d) é a probabilidade de \textbf{d} pertencer a classe
  \textbf{c}. Ou seja, a probabilidade de \textit{``This place is great''} ser
  uma frase positiva ou negativa.
  \item P(d$\vert$c) é a probabilidade da classe \textbf{c} ser \textbf{d}. Ou
  seja, dentre todas as frases negativas ou positivas, a probabilidade de
  uma frase ser \textit{``This place is great''}.
  \item P(c) é a probabilidade da classe \textbf{c}. Ou seja, a frequência que
  frases negativas ou positivas aparecem em nosso \textit{training
  set}.
  \item P(d) é a probabilidade de \textbf{d}. Ou seja, a frequência que
  a frase \textit{``This place is great''} aparece em nosso \textit{training
  set}.
\end{itemize}

Como ambas as Equações terão como divisor P(d), este é removido da equação, tendo como
comparação entre as probabilidades negativas e positivas as Equações
\ref{eq:teoreman} e \ref{eq:teoremap}:
\begin{equation}
\begin{gathered}
P(Negativa|\textit{This place is great})
=
P(\textit{This place is great}|Negativa) \times
P(Negativa)
\label{eq:teoreman}
\end{gathered}
\end{equation}
\begin{equation}
\begin{gathered}
P(Positiva|\textit{This place is great})
=
P(\textit{This place is great}|Positiva) \times
P(Positiva)
\label{eq:teoremap}
\end{gathered}
\end{equation}

Os termos P(Positiva) e P(Negativa) são definidos pela frequência que frases
positivas e negativas aparecem no \textit{training set}, sendo determinados
através das Equações \ref{eq:frasespositivas} e \ref{eq:frasesnegativas}. Neste
caso, \textit{``The food was great''}, \textit{``I love the food here''}, \textit{``This place is
wonderful''} são frases positivas e as demais frases ``\textit{They are horrible!}'' e
``\textit{Forgettable experience}'' são negativas, do conjunto de 5 frases do
\textit{training set}.
\begin{equation}
\begin{gathered}
P(Positiva)
=
\frac{3}{5} = 0,6
\label{eq:frasespositivas}
\end{gathered}
\end{equation}
\begin{equation}
\begin{gathered}
P(Negativa)
=
\frac{2}{5} = 0,4
\label{eq:frasesnegativas}
\end{gathered}
\end{equation}
Porém, a frase \textit{``This place is great''} não existe por completo no
\textit{training set} tornando o termo P(d$\vert$c) da Equação
\ref{eq:teorema} 0 e impossibilitando o cálculo de probabilidade para essa frase.
Neste caso, se faz o uso do \textit{Naive Bayes}, o qual passa a
considerar palavras ao invés de frases completas, solucionando o problema de
frases não encontradas no \textit{training sets}.

\subsection{\textit{Naive Bayes} aplicado ao Teorema}

Bayes como visto, é relacionado com o teorema utilizado para cálculo da
probabilidade, já a palavra \textit{naive}, ou ingênuo, é relacionada uma
outra característica, que é a
independência entre atributos. Esses, para o aprendizado de máquina, são
características da informação que estamos classificando. Por exemplo, ao efetuar uma classificação relacionada com medicina, atributos que seriam considerados poderiam ser histórico de doença, altura da pessoa e peso. No caso da análise de sentimentos, os atributos
são as próprias palavras do texto, ou seja, em sua classificação, ele ignora a
ordem das palavras e somente considera a frequência na qual elas aparecem.
Portanto para o \textit{Naive Bayes} o termo P(\textit{This place is
great}$\vert$ Positiva) visto na Equação \ref{eq:teoremap} é dado pela Equação
\ref{eq:frasespositivas1}:
\begin{equation}
\begin{gathered}
P(\textit{This place is great}|Positiva) = P(\textit{This}|Positiva)
\times P(\textit{place}|Positiva) \\ \times P(\textit{is}|Positiva) \times
P(\textit{great}|Positiva)
\label{eq:frasespositivas1}
\end{gathered}
\end{equation}

A partir da Equação \ref{eq:frasespositivas1} se faz necessário cálcular os
termos P(\textit{This}$\vert$Positiva),
P(\textit{place}$\vert$Positiva), P(\textit{is}$\vert$Positiva), P(\textit{great}$\vert$Positiva).
Aonde que, por exemplo, P(\textit{This}$\vert$Positiva) é a quantidade de vezes
que a palavra \textit{This} foi classificada como positiva em nosso
\textit{training set}, dividido pelo total de palavras classificadas como positiva:

\begin{equation}
\begin{gathered}
P(\textit{This}|Positiva) = \frac{1}{13}
\label{eq:thispositiva}
\end{gathered}
\end{equation}


Da mesma forma, a Equação \ref{eq:thispositiva} é aplicada para as demais
palavras da frase que está sendo classificada obtendo os valores representados
pela Tabela \ref{tab:probabilidadesnb}:

\begin{table}[htb]
\centering
\renewcommand{\arraystretch}{1.5}% Spread rows out...
\begin{tabular}{lll}
\hline

Palavra & Positiva & Negativa \\ \hline
This & \large $\frac{1}{13}$ & \large $\frac{0}{5}$ \\
place & \large $\frac{1}{13}$ & \large $\frac{0}{5}$ \\
is & \large $\frac{1}{13}$ & \large $\frac{0}{5}$ \\
great & \large $\frac{1}{13}$ & \large $\frac{0}{5}$ \\
\end{tabular}
\caption{Tabela de Palavras e Probabilidades.}
\label{tab:probabilidadesnb}
\end{table}

Como algumas palavras não existem em nosso \textit{training set} para
determinadas situações, elas acabam tornando o resultado final da multiplicação
das probabilidades de cada palavra (Equação \ref{eq:frasespositivas1}) como 0,
para evitar que uma única palavra invalide uma frase é utilizado \textit{Laplace smoothing}. Neste, é somado
1 a cada palavra e ao total de palavras, são somadas as quantidades de palavras diferentes do \textit{training set}, que neste caso são
as seguintes 16 palavras distintas: \textit{``The''}, \textit{``food''},
\textit{``was"}, \textit{``great"}, \textit{``They"}, \textit{``are"},
\textit{``horrible!''}, \textit{``I"}, \textit{``love"}, \textit{``here''},
\textit{``This"}, \textit{``place"}, \textit{``is"}, \textit{``wonderful"},
\textit{``Forgettable"}, \textit{``experience"}. Neste caso, aplicando o
\textit{Laplace smoothing} para a Tabela \ref{tab:probabilidadesnb} é obtida a
Tabela \ref{tab:probabilidadesl}.


\begin{table}[htb]
\centering
\renewcommand{\arraystretch}{1.5}% Spread rows out...
\begin{tabular}{lll}
\hline

Palavra & Positiva & Negativa \\ \hline
This & \large $\frac{1 + 1}{13 + 16}$ & \large $\frac{0 + 1}{5 + 16}$ \\
place & \large $\frac{1 + 1}{13 + 16}$ & \large $\frac{0 + 1}{5 + 16}$ \\
is & \large $\frac{1 + 1}{13 + 16}$ & \large $\frac{0 + 1}{5 + 16}$ \\
great & \large $\frac{1 + 1}{13 + 16}$ & \large $\frac{0 + 1}{5 + 16}$ \\
\end{tabular}
\caption{Tabela de Probabilidades - \textit{Laplace smoothing}.}
\label{tab:probabilidadesl}
\end{table}

Utilizando as probabilidades obtidas na Tabela
\ref{tab:probabilidadesl} sob os termos da Equação \ref{eq:frasespositivas1} o
classificador obtem a seguinte Equação:

\begin{equation}
\begin{gathered}
P(Positiva|\textit{This place is great}) = \frac{1 + 1}{13 + 16} \times
\frac{1 + 1}{13 + 16} \times \frac{1 + 1}{13 + 16} \times
\frac{1 + 1}{13 + 16} = 0,000023.
\label{eq:ppositivaplace}
\end{gathered}
\end{equation}

Com o termo P(Positiva $\vert$ \textit{This place is great}) definido através da
Equação \ref{eq:ppositivaplace}, este é aplicado a Equação \ref{eq:teoreman},
junto com o termo P(Positiva) definido através da Equação \ref{eq:frasespositivas}, obtendo a probabilidade da frase
\textit{``This place is great''} ser classificada como positiva através da
Equação \ref{eq:positiva}:

\begin{equation}
\begin{gathered}
P(Positiva|\textit{This place is great})
=
0,000023 \times
0,6 = 0,0000138.
\label{eq:positiva}
\end{gathered}
\end{equation}

Efetuando o mesmo processo para a probabilidade da frase ser negativa, obtemos:
\begin{equation}
\begin{gathered}
P(Negativa|\textit{This place is great})
=
0,0000049 \times
0,4 = 0,00000196.
\label{eq:negativa}
\end{gathered}
\end{equation}

Portanto, a partir dessas duas possibilidade, utilizando o método de
classificação Naive Bayes para efetuar a análise de sentimentos, ele iria
classificar a frase ``This place is great'' como positiva por essa probabilidade
(0,0000138) ser maior que a probabilidade dessa frase ser negativa (0,00000196).



\section{\textit{VADER}}

O \ac{VADER} é um dicionário e classificador de sentimentos que se baseia em
regras, portanto, um método de classificação simbólico. Ele é especialmente
ajustado para funcionar em redes sociais aonde temos um contexto vago e pouca
quantidade de texto, levando em consideração
gírias e emoticons, nesse contexto, ele é extremamente eficaz, podendo se
comparar a classificação feita por humanos \cite{conf/icwsm/HuttoG14}.

A classificação do sentimento é feita através da separação da frase em palavras
e para cada palavra da frase é atribuída uma pontuação de intensidade em uma escala de -4 até +4, como por exemplo, a palavra \textit{great} tem a
intensidade de 3.1 e \textit{horrible} -2.5. Essa pontuação é obtida
através de um dicionário construído utilizando o método de \textit{``wisdom of
the crowd''} aonde um grupo de pessoas atribuiu os valores de intensidade para
cada palavra ao invés uma única pessoa especializada ou uma classificação
automática através de estatística. Por
exemplo, a frase \textit{``This place is great''} seria classificada da seguinte forma:


\begin{table}[htb]
\centering
\begin{tabular}{l|l|l|l|l|l|l}
Palavra         & \textit{This}        & \textit{place} & \textit{is}      &
\textit{great}
\\
%Correta: & Substantivo & Verbo  & Artigo & Substantivo & Preposição &
% Substantivo \\
Intensidade:   &  &   &  & 3,1
\end{tabular}
\label{my-label}
\end{table}

As palavras \textit{``This''}, \textit{``place''} e \textit{``is''},
respectivamente ``Esse'', ``lugar'' e ``é'' não existem no dicionário pois não
expressam sentimentos e portanto são desconsideradas. Após, ele
faz uso do seguinte conjunto de regras para inferir a intensidade do
sentimento:
\begin{itemize}
\item A primeira regra verifica quando uma palavra com pontuação atribuída, ou
seja, uma palavra que expressa sentimentos, é escrita em letras maiúsculas.
  Neste caso, é aumentada a magnitude da intensidade do sentimento sem modificar
  a orientação semântica, somando 0,733 a intensidade do
  sentimento caso este tenha intensidade maior que 0 ou subtraído
  este mesmo valor caso este tenha intensidade menor que 0.
  \begin{table}[htb]
	\centering
	\begin{tabular}{l|l|l|l|l|l|l}
	Palavra         & \textit{This}        & \textit{place} & \textit{is}      &
	\textit{GREAT}
	\\
	%Correta: & Substantivo & Verbo  & Artigo & Substantivo & Preposição &
	% Substantivo \\
	Intensidade:   &  &   &  & 3,1 \textrightarrow 3,833
	\end{tabular}
	\label{my-label}
   \end{table}

\item Após, verifica se alguma das três palavras anteriores a uma palavra com
pontuação atribuída é um advérbio intensificador. Neste caso, estes impactam a
intensidade do sentimento aumentando ou diminuindo 0,293 conforme o advérbio. 
  \begin{table}[htb]
	\centering
	\begin{tabular}{l|l|l|l|l|l|l}
	Palavra         & \textit{This}        & \textit{place} & \textit{is}      &
	\textit{incredibly} & \textit{great}
	\\
	%Correta: & Substantivo & Verbo  & Artigo & Substantivo & Preposição &
	% Substantivo \\
	Intensidade:   &  &   &  & Advérbio & 3,1 \textrightarrow 3,393 
	
	\end{tabular}
	\label{my-label}
   \end{table}
   
   
  \begin{table}[htb]
	\centering
	\begin{tabular}{l|l|l|l|l|l|l}
	Palavra         & \textit{This}        & \textit{place} & \textit{is}      &
	\textit{somewhat} & \textit{great}
	\\
	%Correta: & Substantivo & Verbo  & Artigo & Substantivo & Preposição &
	% Substantivo \\
	Intensidade:   &  &   &  & Advérbio & 3,1 \textrightarrow 2,807
	\end{tabular}
	\label{my-label}
   \end{table}
   

\item A terceira regra, verifica se a frase contém a palavra \textit{``but''}
(mas).
Caso encontrada, essa palavra indica uma troca no sentimento da frase expressa aonde
que o texto seguinte a ela expressa um sentimento mais dominante. Neste caso,
o método multiplica a intensidade dos sentimentos expressos até a palavra
\textit{``but''} por 0,5 e os sentimentos expressos após a palavra
\textit{``but''} são multiplicados por 1,5.

   
   \begin{table}[!htbp]
	\centering
	\begin{tabular}{l|l|l|l|l|l|l|l|l}
	Palavra         & \textit{Great} & \textit{place}      & \textbf{\textit{but}}
	& \textit{today} & \textit{the}      &
	\textit{food} & \textit{was}      & \textit{horrible}
	\\
	%Correta: & Substantivo & Verbo  & Artigo & Substantivo & Preposição &
	% Substantivo \\
	Intensidade: & 3,1 \textrightarrow 1,55  &   &  &  & & & &  -2,5
	\textrightarrow -3,75
	\end{tabular}
	\label{my-label5}
   \end{table}

\item Também, verifica se a frase possui pontos de exclamação (!). Estes,
aumentam a magnitude da intensidade sem modificar a orientação semântica,
somando 0,292 a cada ponto de exclamação considerando um máximo de 4 pontos de
exclamação.

 \begin{table}[!htbp]
	\centering
	\begin{tabular}{l|l|l|l|l|l|l}
	Palavra         & \textit{This}        & \textit{place} & \textit{is}      &
	\textit{great!}
	\\
	%Correta: & Substantivo & Verbo  & Artigo & Substantivo & Preposição &
	% Substantivo \\
	Intensidade:   &  &   &  & 3,1 \textrightarrow 3,392
	\end{tabular}
	\label{my-label}
   \end{table}
   

\item Por fim, a última regra examina as três palavras anteriores,
identificando 90\% dos casos aonde uma negação inverte a polaridade de um texto.
Quando encontrada uma negação na frase, é multiplicada a intensidade de cada
palavra que possui sentimento por -0,74.

 \begin{table}[!htbp]
	\centering
	\begin{tabular}{l|l|l|l|l|l|l}
	Palavra         & \textit{This}        & \textit{place} & \textit{wasn't}     
	&
	\textit{great}
	\\
	%Correta: & Substantivo & Verbo  & Artigo & Substantivo & Preposição &
	% Substantivo \\
	Intensidade:   &  &   & Negação & 3,1 \textrightarrow 2,294
	\end{tabular}
	\label{my-label}
   \end{table}


\end{itemize}

Após ter a pontuação de intensidade, o método normaliza essa pontuação através
da Equação \ref{eq:vader}:

\begin{equation}
\begin{gathered}
\text{Pontuação Normalizada}
=
\frac{\text{Pontuação}}{\sqrt{\text{Pontuação}^2 + 15}}
\label{eq:vader}
\end{gathered}
\end{equation}

A frase \textit{``This place is great''} aonde somente a palavra
\textit{great} possui pontuação e essa tem o valor de 3,1, será atribuida
pontuação final de 0,6249 através da Equação \ref{eq:vaderscore}.  Caso essa
pontuação fosse menor que -1 ou maior que 1, essa seria limitada a -1 ou 1
respectivamente. Para o \ac{VADER}, são consideradas frases negativas aquelas
com pontuação de -1 até 0,5, frases neutras aquelas com pontuação de -0,5 até
0,5 e frases positivas aquelas com pontuação 0,5 até 1, portanto a frase
\textit{``This place is great''} com pontuação 0,6249 seria classificada como
positiva.

\begin{equation}
\begin{gathered}
\text{Pontuação Normalizada}
=
\frac{\text{3,1}}{\sqrt{\text{3,1}^2 + 15}} = 0,6249
\label{eq:vaderscore}
\end{gathered}
\end{equation}
%\chapter{Frameworks}
\label{cap:Frameworks}

A implementação de um software de análise de sentimentos pode ser feita através
de um software escrito a partir de seu início, utilizando como base os métodos
já existentes ou também pode ser feito utilizando \textit{frameworks} já
disponíveis. Métodos estatísticos como o Naive Bayes analisado no Capítulo
\ref{cap:Classificadores} podem ser encontrados tanto em implementações de
aprendizado de máquina, quanto implementações para voltadas para o \ac{NLP}.
Já métodos simbólicos, por suas implementações serem específicas para a o
\ac{NLP} ou análise de sentimentos, somente encontramos implementações destes em
\textit{frameworks} de \ac{NLP}.

O método selecionado para a análise de sentimentos, \ac{VADER}, é
disponibilizado como um \textit{package} Python e também através do
\textit{framework} \ac{NLTK}, o qual é considerado um \textit{kit} de
ferramentas não só para análise de sentimentos mas como o \ac{NLP} como um todo.

Por trazer diversas outras implementações que possam a vir facilitar a
identificação de padrões de sentimentos, foi escolhido utilizar o \ac{NLTK} para
acessar o \ac{VADER} ao invés de somente a utilização do próprio
\textit{package} desse.

O \ac{NLTK} é um \textit{framework open source} para Python
criado em 2001 na Universidade de Pensilvânia atualmente utilizado por mais de
30 universidades em diversos
países\footnote{\url{https://docs.google.com/document/d/1eYubSwLkpB7ZgfQVxxAwgsmAqS__BRfbMyP9qV6ngD8/edit}}.
Esse apresenta tanto métodos estatísticos, como \ac{MaxEnt} e Naive Bayes como métodos simbólicos, contendo
mais de 50 dicionários e modelos já treinados como \textit{Sentiment Polarity Dataset Version 2.0}, conjunto de dados já classificados que contém mais de 1000 filmes avaliados de forma positiva e 1000 filmes avaliados de forma negativa, \textit{SentiWordNet},
um dicionário com as palavras extraídas do WordNet já classificadas em
positividade, negatividade e objetividade e \ac{VADER} o qual foi selecionado
para aplicação da análise de sentimentos.

Além da execução de análise de sentimentos, o \textit{framework} executa outras
tarefas pertencentes ao \ac{NLP} como o reconhecimento de entidade mencionada e
análise léxica.



\chapter{Criação de Base de Dados}
\label{cap:banco}
Este capítulo tem como objetivo descrever sucintamente a rede social Reddit.
Após são apresentados os tópicos que foram selecionados para a análise de
sentimentos. Por fim, é apresentada a ferramenta desenvolvida para a extração
dos comentários destes tópicos e para a criação da base.
\section{Rede Social Reddit}
\label{cap:Reddit}

O \textit{website} Reddit teve seu início em 2005 como um agregador de
conteúdo e, atualmente, é o vigésimo terceiro \textit{website} mais acessado na
internet e o sétimo mais acessado nos Estados Unidos da América \cite{alexa}.
Os usuários do Reddit podem enviar \textit{links} com conteúdos externos
ao Reddit ou ainda mensagens de texto. A partir desse conteúdo, os seus
usuários podem votar para cima (\textit{upvote}) ou para baixo \textit{downvote},
influenciando na posição do conteúdo no \textit{website}. Além de votar no conteúdo, seus usuários podem enviar comentários como
forma de expressar sua opinião (Figura \ref{fig:reddit}).

\begin{figure}[!htbp]
\centering
\includegraphics[height=300px]{imagens/reddit.png}
\caption{\textit{Website Reddit}:  As flechas demarcadas permitem efetuarmos
\textit{upvotes} ou \textit{downvotes}.}
\label{fig:reddit}
\end{figure}

O conteúdo do Reddit é distribuído em \textit{subreddits} que funcionam como
comunidades. Os usuários podem se inscrever nesses
\textit{subreddits}, recebendo as atualizações na sua página inicial, sendo
que dentre esses \textit{subreddits}, destacam-se:


\begin{itemize}
  \item \textit{/r/AskReddit}: esse \textit{subreddit} é utilizado para fazer
  perguntas gerais para outros usuários do Reddit. Esse \textit{subreddit}
  possui aproximadamente 16.941.544 inscritos.
  \item \textit{/r/worldnews}: esse \textit{subreddit} possui as notícias de
  todo o mundo, contando com aproximadamente 16.570.606 inscritos.
  \item \textit{/r/IAmA}: IAmA é um estilização de 'I am a' ('Eu sou um'):
  a partir desse \textit{subreddit} os usuários podem fazer perguntas ao criador
  de um determinado tópico. Esse \textit{subreddit} possui aproximadamente
  16.990.161 inscritos.
\end{itemize}

% Dentre esses \textit{subreddits} podemos destacar alguns dos tópicos mais
% acessados no ano de 2016:
% 
% \begin{itemize}
%   \item \textit{/r/IAmA} - \textit{We're NASA scientists \& exoplanet experts.
%   Ask us anything about today's announcement of seven Earth-size planets
%   orbiting TRAPPIST-1!} - Tópico de perguntas e respostas com cientistas da
%   NASA após a descoberta dos planetas que orbitavam a estrela TRAPPIST-1.
%   \item \textit{/r/IAmA} - \textit{I’m Bill Gates, co-chair of the Bill \&
%   Melinda Gates Foundation. Ask Me Anything.} - Tópico de perguntas e respostas com Bill Gates.
%   \item \textit{/r/worldnews} - \textit{Fidel Castro is dead at 90.} - Link para
%   anúncio da morte de Fidel Castro.
%   \item \textit{/r/AskReddit} - \textit{[Serious]South Koreans of Reddit, how
%   did they teach you about the existence of North Korea in School when you were
%   young?serious replies only} - Tópico perguntando para os usuários sul coreanos
%   como que foi ensinado para eles sobre a existência da Coreia do Norte.
% \end{itemize}

% A identificação de padrões de sentimentos expressos por determinados grupos
% dessa comunidade, se faz útil visto que a partir dessa avaliação é possível construir
% ferramentas que apoiam decisões tanto de um ponto de vista político, como por
% exemplo, entender qual é a opinião sobre um determinado assunto de um conjunto
% de eleitores, tanto quanto um ponto de vista de negócios, para entender qual a
% opinião dos consumidores de um produto, ou de seu competidor, a respeito de um
% determinado assunto.

\section{Extração de Dados}
\label{cap:Extracao}

Para a extração comentários do \textit{website} Reddit e para a criação da base
foi desenvolvido um \textit{crawler} ou robô de navegação. Esse robô foi
implementado na linguagem Java e tem como objetivo a navegação automática no
conteúdo do \textit{website} Reddit, extraindo os dados e comentários referentes a um
determinado tópico. Após, os dados são armazenados em banco de dados MySQL
\cite{Widenius:2002:MRM:560480}.
Na Figura \ref{fig:crawler} tem-se a arquitetura do \textit{software}
desenvolvido. O código deste encontra-se no Anexo A.

\newpage

\begin{figure}[htbp]
\centering
\includegraphics[height=225px]{imagens/arquitetura.png}
\caption{Arquitetura do \textit{Crawler}}
\label{fig:crawler}
\end{figure}

A partir de um \textit{link} para um tópico, o robô efetua uma busca e a
extração dos dados relacionados a esse tópico. Para tanto, foi utilizada a
\ac{API} do Reddit, onde, inicialmente envia-se uma requisição utilizando o
sufixo ``.json'' (Por exemplo:
\textit{\url{https://www.reddit.com/r/iama.json}}) e, a partir dessa requisição,
o \textit{website} retorna um objeto \ac{JSON}. Uma vez que o \ac{JSON}
retornado pelo \textit{website} possui 68 campos e que esses não se encontram
documentados, utilizou-se o \textit{website}
jsonschema2pojo\footnote{http://www.jsonschema2pojo.org/ - O \textit{website}
jsonschema2pojo tem como objetivo a conversão de um esquema \ac{JSON} em \ac{POJO}, permitindo o
\textit{download} da classe para a utilização.} para converter o
JSON retornado em um objeto \ac{POJO}.

Após, foi utilizado o \textit{framework} Hibernate
\cite{Iverson:2004:HJD:1044870} para a criação do banco de dados e para a
persistência dos dados. O Hibernate é um \textit{framework} de
mapeamento objeto-relacional que tem como objetivo representar tabelas do banco
de dados através de classes, ou seja, esse tem como principal característica a
transformação das classes em Java para tabelas em um banco de dados relacional.
No caso deste trabalho, ele é responsável pela criação das tabelas \textit{RedditPost} e
\textit{RedditThread}, relacionadas respectivamente com os comentários e o
tópico em questão. 

\section{Tópicos Selecionados}

Para análise de sentimentos e para comparação dos resultados obtidos, foram
selecionados 15 tópicos, sendo que esses tópicos são os que
apresentam maior número de comentários no último ano. Destaca-se que os 15
tópicos encontram-se distribuídos em diferentes assuntos, que são: cenário
político nacional, cenário político internacional e tópicos diversos:

No que diz respeito a tópicos relacionados com ao cenário político nacional, os
tópicos escolhidos foram:
\sloppy
\begin{itemize}
  \item
  \textit{Brazil Seeks To Copy U.S. Gun Culture ``to allow embattled
  citizens the right to defend themselves from
  criminals''}: esse tópico encontra-se disponível em 
  \url{https://www.reddit.com/r/worldnews/comments/36ny58/brazil_blogger_known_for_reporting_on_corruption/}
  e refere-se a intenção do Brasil em adotar a cultura de porte de armas dos
  Estados Unidos da América.
  \item
  \textit{Brazil descends into chaos as Olympics looms}: esse tópico encontra-se disponível em 
  \url{https://www.reddit.com/r/worldnews/comments/4bqcc3/brazil_descends_into_chaos_as_olympics_looms/}
  e refere-se ao caos ocorrido nas Olímpiadas de 2016 que foram realizadas no
  Brasil.
  \item
  \textit{Plane carrying Brazil Supreme Court judge crashes into sea}: esse tópico encontra-se disponível em
  \url{https://www.reddit.com/r/worldnews/comments/5oyz3b/plane_carrying_brazil_supreme_court_judge_crashes/}
  e refere-se a queda do avião no qual o ministro Teori Zavascki estava abordo.
  \item
  \textit{Brazil passes Internet governance Bill: Brazil has made history with
  the approval of a post-Snowden Bill which sets out principles, rights and
  guarantees for Internet users.}: esse tópico encontra-se disponível em
  \url{https://www.reddit.com/r/worldnews/comments/21f3as/brazil_passes_internet_governance_bill_brazil_has/}
  e refere-se a aprovação do Marco Civil da Internet.
  \item
  \textit{FIFA generated more than \$4 billion in sales from the 2014 World Cup,
  and is Giving Brazil \$100 Million After The Country Spent \$15 Billion On The
  World Cup}: esse tópico encontra-se disponível em
  \url{https://www.reddit.com/r/worldnews/comments/2t65ql/fifa_generated_more_than_4_billion_in_sales_from/}
  e refere-se a diferença entre o que foi gasto e o que foi arrecadado pelo
  Brasil na Copa do Mundo de 2014.
 
\end{itemize}

Já os tópicos escolhidos que se referem a política internacional são:
\begin{itemize}
  \item
  \textit{2.6 terabyte leak of Panamanian shell company data reveals "how a
  global industry led by major banks, legal firms, and asset management companies
  secretly manages the estates of politicians, Fifa officials, fraudsters and
  drug smugglers, celebrities and professional
  athletes."}.: esse tópico encontra-se disponível em
  \url{https://www.reddit.com/r/worldnews/comments/4d75i7/26_terabyte_leak_of_panamanian_shell_company_data/}
  e refere-se ao vazamento dos documentos confidenciais de uma
  sociedade de advogados panamenha. Esses documentos apresentam informações
  detalhadas de empresas localizadas em paraísos fiscais.
  \item
  \textit{Fidel Castro is dead at
  90.}: esse tópico encontra-se disponível em
  \url{https://www.reddit.com/r/worldnews/comments/5exz2e/fidel_castro_is_dead_at_90/}
  e se refere a morte do presidente de Cuba, Fidel Castro.
  
  \item
  \textit{Donald Trump to strip all funding from State Dept team promoting
  women's rights around the world - Leaked plan comes as First Daughter Ivanka
  defends her father's record with women}: esse tópico encontra-se disponível em
  \url{https://www.reddit.com/r/worldnews/comments/67ivae/donald_trump_to_strip_all_funding_from_state_dept/}
  e refere-se a decisão do presidente dos Estados Unidos da América, Donald
  Trump, em remover fundos de promoção ao direito das mulheres.
  
  \item
  \textit{Manchester Arena 'explosions': Two loud bangs heard at MEN Arena}:
  esse tópico encontra-se disponível em
  \url{https://www.reddit.com/r/worldnews/comments/6cqdye/manchester_arena_explosions_two_loud_bangs_heard/}
  e refere-se ao atentado terrorista ocorrido na Manchester Arena (Inglaterra)
  em 23 de Maio de 2017.
  
  \item
  \textit{Sweden asks the U.S. to explain Trump comment on
  Sweden}: esse tópico encontra-se disponível em
  \url{https://www.reddit.com/r/worldnews/comments/5uzetf/sweden_asks_the_us_to_explain_trump_comment_on/}
  e se refere aos comentários feitos do presidente dos Estados Unidos da
  América, Donald Trump, sobre a Suécia.
  
  \item\textit{“Canada will welcome you,” Trudeau invites refugees as Trump bans
  them}: esse tópico encontra-se disponível em
  \url{https://www.reddit.com/r/worldnews/comments/5qqa51/canada_will_welcome_you_trudeau_invites_refugees/}
  e refere-se a declaração do primeiro ministro canadense sobre decisão de
  receber refugiados. Neste declaração, o primeiro ministro canadense afirma que
  os refugiados serão bem-vindos no Canadá.
\end{itemize}

Por fim, os tópicos selecionados que abordam assuntos diversos foram:
\begin{itemize}
  \item
  \textit{I’m Bill Gates, co-chair of the Bill \& Melinda Gates Foundation. Ask
  Me Anything.}: esse tópico encontra-se disponível em
  \url{https://www.reddit.com/r/IAmA/comments/5whpqs/im_bill_gates_cochair_of_the_bill_melinda_gates/}
  e refere-se as respostas às perguntas que foram feitas ao fundador da
  Microsoft, Bill Gates.
  \item
  \textit{Hey, it's Lars from Metallica. AMA}: esse tópico encontra-se disponível em
  \url{https://www.reddit.com/r/IAmA/comments/1wl9ic/hey_its_lars_from_metallica_ama/}.
  Esse tópico apresenta as respostas às perguntas realizadas ao vocalista da
  banda de rock Metallica, James Hetfield.
  \item
  \textit{I'm the CEO of Renault and Nissan and we're making autonomous driving
  vehicles happen by 2020. Ask me anything!}: esse tópico encontra-se disponível em
  \url{https://www.reddit.com/r/IAmA/comments/2s7obx/im_the_ceo_of_renault_and_nissan_and_were_making/}
  e refere-se as respostas às perguntas realizadas ao diretor executivo da Renault e Nissan,
  Carlos Ghosn.
  
  \item
  \textit{I am Julian Assange founder of WikiLeaks -- Ask Me Anything}: esse
  tópico encontra-se disponível em
  \url{https://www.reddit.com/r/IAmA/comments/5n58sm/i_am_julian_assange_founder_of_wikileaks_ask_me/}
  e refere-se as respostas às perguntas realizadas ao Julian Assange, fundador do
  \textit{WikiLeaks}.
  
\end{itemize}


Destaca-se que para a criação da base de dados, somente foram extraídos
comentários em resposta ao tópico em questão, comentários em resposta a outros
comentários foram desconsiderados uma vez que esses podem não estar relacionados
diretamente ao tópico em questão, tornando inválida ou prejudicando a
análise de sentimento.


\chapter{Conclusão Parcial}
\label{cap:conclusao}
A \ac{NLP} tem como objetivo a análise de linguagem natural, seja
essa escrita ou falada. Dentre diversas tarefas que ela executa, uma delas é a
análise de sentimentos, a qual se faz útil visto que cada vez
mais as pessoas se comunicam através de redes sociais, gerando um grande volume
de dados. A análise e quantificação da opinião expressa por esses dados, seja
por fins políticos, comerciais ou quaisquer outros, se torna díficil devido a
grande quantidade de dados.

Observou-se que os métodos mais utilizados para análise de sentimentos são o
Método de Naive Bayes (estatístico) e o Método de \ac{VADER} (simbólico). Dentre
estes, optou-se pela utilização do método de \ac{VADER}, uma vez que de acordo
com a literatura, esse apresenta um desempenho superior ao método de Naive
Bayes. De fato, esse mostrou-se superior na análise de sentimentos nas avaliações de
produtos da Amazon, editoriais do New York Times e mais importante, na análise
de \textit{Tweets} da rede social Twitter \cite{SentimentinSocialMedia}. A
justificativa para isso, se dá ao fato de métodos estatísticos necessitarem de um \textit{training set}
especializado para obter resultados similares ou superiores aos métodos
simbólicos. 

Além disso, optou-se pela utilização do método de \ac{VADER}, devido
ao fato deste não necessitar da criação de um \textit{training set} específico
para a análise de sentimentos. A necessidade da criação de um \textit{training
set} específico para cada tema inviabilizaria o desenvolvimento deste trabalho,
uma vez que neste serão analisados 15 tópicos com temas distintos.

Para implementação do método \ac{VADER} optou-se pela biblioteca \ac{NLTK}. A
utilização da biblioteca \ac{NLTK} permitirá a utilização futura de outros
métodos de \ac{NLP}.
 
Por fim, se fez necessária a criação de uma base de dados para armazenar os
tópicos e comentários disponibilizados pela rede social Reddit. Para isso, foi
desenvolvido um \textit{crawler} (robô) que é responsável por extrair os
comentários relacionados a um tópico na rede social Reddit e armazenar em um
banco de dados MySQL. Esse robô foi desenvolvido na linguagem Java e utiliza a
API da rede social Reddit para extrair as informações de um tópico. Após, ele
utiliza o \textit{framework} Hibernate para armazenar os dados extraídos em uma
base de dados relacional em um banco MySQL.

Na segunda etapa deste trabalho, será utilizado o \textit{framework} \ac{NLTK}
para efetuar a análise de sentimento sobre a base de dados criada com o objetivo
de identificar padrões de sentimentos entre usuários e comunidades.

\section{Atividades e Cronograma}

Na Tabela \ref{tab:tcc1} tem-se o cronongrama das atividades realizadas durante
o TCC I.
\begin{enumerate}
\item Estudo de algoritmos para o processamento de texto e também análise de
sentimentos.
\item Análise das ferramentas já existentes.
\item Análise da API do Reddit.
\item Construção de um software para extração dos dados da API.
\item Extração e criação da base de dados.
\item Redação da monografia TCC I.
\item Apresentação TCC I.
\end{enumerate}

\renewcommand{\arraystretch}{2}
\newcolumntype{Y}{>{\centering\arraybackslash}X}
\begin{table}[!htb]
\begin{tabularx}{0.9\textwidth}{Y|Y|Y|Y|Y|Y|Y|Y|Y|Y|Y|}
& \multicolumn{2}{|c|}{Mar} & \multicolumn{2}{|c|}{Abr} &
\multicolumn{2}{|c|}{Mai} & \multicolumn{2}{|c|}{Jun} &
\multicolumn{2}{|c|}{Jul}
\\
\midrule
1 & \cellcolor{black!80} & \cellcolor{black!80} & & & & & & & & \\
2 &  & \cellcolor{black!80} & \cellcolor{black!80} & & & & & & &\\
3 &  &  &  & \cellcolor{black!80} & & & & & &\\
4 &  &  &  &  & \cellcolor{black!80} & & & &  &\\
5 &  &  &  &  &  & \cellcolor{black!80} & \cellcolor{black!80} & & &\\
6 &  & \cellcolor{black!80}  & \cellcolor{black!80}  &  \cellcolor{black!80} & 
\cellcolor{black!80} & \cellcolor{black!80} & \cellcolor{black!80} & \cellcolor{black!80} &  &\\
7 &  &  &  &  &  & & & & \cellcolor{black!80} &\\
\end{tabularx}

\caption{Cronograma do TCC I.}
\label{tab:tcc1}
\end{table}

Já na Tabela \ref{tab:tcc2} tem-se as atividades a serem desenvolvidas no TCC
II:

\begin{enumerate}
\item Implementação do software de Processamento de Linguagem Natural para a
análise de sentimentos na base de dados criada.
\item Análise dos resultados obtidos.
\item Redação da monografia TCC II.
\item Apresentação do TCC II.
\end{enumerate}

\newcolumntype{Y}{>{\centering\arraybackslash}X}
\begin{table}[!htb]
\begin{tabularx}{0.9\textwidth}{Y|Y|Y|Y|Y|Y|Y|Y|Y|Y|Y|}
& \multicolumn{2}{|c|}{Ago} & \multicolumn{2}{|c|}{Set} &
\multicolumn{2}{|c|}{Out} & \multicolumn{2}{|c|}{Nov} &
\multicolumn{2}{|c|}{Dez}
\\
\midrule
1 & \cellcolor{black!80} & \cellcolor{black!80} & \cellcolor{black!80} &
\cellcolor{black!80} & & & & & & \\
2 &  & & & \cellcolor{black!80} & \cellcolor{black!80} & \cellcolor{black!80} &
& & &\\
3 &  & \cellcolor{black!80} & \cellcolor{black!80} & \cellcolor{black!80} &
\cellcolor{black!80} & \cellcolor{black!80} & \cellcolor{black!80}
& \cellcolor{black!80} & &\\
4 &  &  &  &  &  & & & & \cellcolor{black!80} &\\
\end{tabularx}

\caption{Cronograma do TCC II.}
\label{tab:tcc2}
\end{table}

% No Capítulo \ref{cap:Processamento} foram introduzidos dois tipos de métodos
% distintos para o \ac{NLP}, métodos simbólicos e
% métodos estatísticos, os quais foram estudados para a análise de sentimentos
% através do Capítulo \ref{cap:Classificadores}.
% 
% Através do Capítulo \ref{cap:Classificadores}, foram comparados um método
% simbólico e outro método estatístico a fim de se determinar qual apresenta melhor performance na análise de sentimentos
% aplicada em uma rede social, sendo que a literatura apontou que o método mais
% assertivo é o método \ac{VADER}, o qual está disponível através do
% \textit{framework} \ac{NLTK}.
% 
% Já no Capítulo \ref{cap:banco}, foi apresentada como funciona a rede social
% Reddit, os tópicos selecionados para a análise de sentimentos, e por fim foi
% apresentado a forma na qual esses tópicos serão extraídos para população da base
% de dados.
% 
% A partir das informações demonstradas através deste, deverá ser possível criar
% um \textit{software} que efetue a análise de dados utilizando o \ac{VADER},
% através do \textit{framework} \ac{NLTK}, aplicada nos tópicos demonstrados no
% Capítulo \ref{cap:banco}.



%\chapter{Criação de Base de Dados}
\label{cap:banco}
Este capítulo tem como objetivo descrever a rede social Reddit e apresentar os
tópicos que foram selecionados para a análise de sentimentos. Após, é
demonstrada a ferramenta desenvolvida para a extração dos comentários destes tópicos e para a criação da base. Por fim, é
apresentada a validação da ferramenta desenvolvida.
\section{Rede Social Reddit}
\label{cap:Reddit}

O \textit{website} Reddit foi criado por Alexis Ohanian e Steve Huffman e teve
seu início em 2005 como um agregador de conteúdo e, atualmente, é o vigésimo terceiro \textit{website} mais acessado na
internet e o sétimo mais acessado nos Estados Unidos da América \cite{alexa}.
Os usuários do Reddit podem enviar \textit{links} com conteúdos externos
ao Reddit ou ainda mensagens de texto. A partir desse conteúdo, os seus
usuários podem votar para cima (\textit{upvote}) ou para baixo \textit{downvote},
influenciando na posição do conteúdo no \textit{website}. Além disso, seus
usuários podem enviar comentários como forma de expressar sua opinião (Figura \ref{fig:reddit}).


\begin{figure}[!htbp]
\centering
\includegraphics[height=300px]{imagens/reddit.png}
\caption{\textit{Website Reddit}:  As flechas demarcadas permitem efetuarmos
\textit{upvotes} ou \textit{downvotes}.}
\label{fig:reddit}
\end{figure}

O conteúdo do Reddit encontra-se distribuído em \textit{subreddits} que
funcionam como comunidades. Os usuários podem se inscrever nesses
\textit{subreddits}, recebendo as atualizações na sua página inicial, sendo
que dentre esses \textit{subreddits}, destacam-se:


\begin{itemize}
  \item \textit{/r/AskReddit}: esse \textit{subreddit} é utilizado para fazer
  perguntas gerais para outros usuários do Reddit. Esse \textit{subreddit}
  possui aproximadamente 16.941.540 inscritos.
  \item \textit{/r/worldnews}: esse \textit{subreddit} possui as notícias de
  todo o mundo, possuindo aproximadamente 16.570.600 inscritos.
  \item \textit{/r/IAmA}: IAmA é um estilização de \textit{'I am a'} ('Eu sou
  um'):
  a partir desse \textit{subreddit} os usuários podem fazer perguntas ao criador
  de um determinado tópico. Esse \textit{subreddit} possui aproximadamente
  16.990.160 inscritos.
\end{itemize}

% Dentre esses \textit{subreddits} podemos destacar alguns dos tópicos mais
% acessados no ano de 2016:
% 
% \begin{itemize}
%   \item \textit{/r/IAmA} - \textit{We're NASA scientists \& exoplanet experts.
%   Ask us anything about today's announcement of seven Earth-size planets
%   orbiting TRAPPIST-1!} - Tópico de perguntas e respostas com cientistas da
%   NASA após a descoberta dos planetas que orbitavam a estrela TRAPPIST-1.
%   \item \textit{/r/IAmA} - \textit{I’m Bill Gates, co-chair of the Bill \&
%   Melinda Gates Foundation. Ask Me Anything.} - Tópico de perguntas e respostas com Bill Gates.
%   \item \textit{/r/worldnews} - \textit{Fidel Castro is dead at 90.} - Link para
%   anúncio da morte de Fidel Castro.
%   \item \textit{/r/AskReddit} - \textit{[Serious]South Koreans of Reddit, how
%   did they teach you about the existence of North Korea in School when you were
%   young?serious replies only} - Tópico perguntando para os usuários sul coreanos
%   como que foi ensinado para eles sobre a existência da Coreia do Norte.
% \end{itemize}

% A identificação de padrões de sentimentos expressos por determinados grupos
% dessa comunidade, se faz útil visto que a partir dessa avaliação é possível construir
% ferramentas que apoiam decisões tanto de um ponto de vista político, como por
% exemplo, entender qual é a opinião sobre um determinado assunto de um conjunto
% de eleitores, tanto quanto um ponto de vista de negócios, para entender qual a
% opinião dos consumidores de um produto, ou de seu competidor, a respeito de um
% determinado assunto.

\section{Extração de Dados}
\label{cap:Extracao}

Para a extração de comentários do \textit{website} Reddit e para a criação da
base foi desenvolvido um \textit{crawler} ou robô de navegação. Esse robô foi
implementado na linguagem de programação Java e tem como objetivo a navegação
automática no conteúdo do \textit{website}, extraindo os dados e comentários referentes a um
determinado tópico. Após, os dados são armazenados em uma base de dados criada
em um banco de dados MySQL \cite{Widenius:2002:MRM:560480}.
Na Figura \ref{fig:crawler} tem-se a arquitetura do \textit{software}
desenvolvido. O código deste encontra-se no CD em anexo.

\begin{figure}[htbp]
\centering
\includegraphics[height=225px]{imagens/arquitetura.png}
\caption{Arquitetura do \textit{Crawler}}
\label{fig:crawler}
\end{figure}

A partir do \textit{link} de um tópico, o robô efetua uma busca e a
extração dos dados relacionados a esse tópico. Para tanto, foi utilizada a
\ac{API} do Reddit, onde, inicialmente envia-se uma requisição utilizando o
sufixo ``.json'' (Por exemplo:
\textit{\url{https://www.reddit.com/r/iama.json}}) e, a partir dessa requisição,
o \textit{website} retorna um objeto \ac{JSON}. Uma vez que o \ac{JSON}
retornado possui 68 campos e que esses não se encontram
documentados, utilizou-se o \textit{website}
\textit{jsonschema2pojo}\footnote{http://www.jsonschema2pojo.org/ - O
\textit{website} jsonschema2pojo tem como objetivo a conversão de um esquema \ac{JSON} em
\textit{Plain Old Java Objects}\ac{POJO}, permitindo o \textit{download} da classe para a utilização.} para converter o
JSON retornado em um \textit{Plain Old Java Objects} (\ac{POJO}).

Após, foi utilizado o \textit{framework} Hibernate
\cite{Iverson:2004:HJD:1044870} para a criação do banco de dados e para a
persistência dos dados. O Hibernate é um \textit{framework} de
mapeamento objeto-relacional que tem como principal objetivo 
representar as tabelas de um banco de dados através de classes, ou seja, esse \textit{framework} tem como principal
característica a transformação das classes em Java em tabelas em um banco de
dados relacional.
No caso deste trabalho, ele é responsável pela criação das tabelas \textit{RedditPost} e
\textit{RedditThread}, relacionadas, respectivamente, com os comentários e o
tópico em questão. 

\section{Validação da Implementação}

Para validação da implementação desenvolvida foi selecionado o tópico:
\textit{``Canada will welcome you, Trudeau invites refugees as Trump bans
them''.}
A partir desse tópico, foram extraídos todos os comentários, sendo que
esses foram avaliados de forma manual. Após, esses comentários foram processados
utilizando-se a \ac{NLTK}, obtendo uma assertividade de 56\%.

Destaca-se que uma grande parte dos comentários que foram identificados de forma
incorreta, apresentavam a irônia como característica, como por exemplo, o
comentário \textit{``lol Good Luck Canada''}. A identificação da ironia através
da leitura textual é difícil até mesmo para um ser-humano, uma vez que o
texto pode não apresentar sinais de humor. De fato, a ironia para a análise de
sentimentos é considerado um tópico problemático sendo alvo de diversos estudos \cite{DBLP:conf/lrec/StranisciBFP16}.

Observou-se que nos demais comentários que foram identificados de forma
incorreta, o sentimento expresso não se encontrava no dicionário de
sentimentos padrão do \ac{VADER}. Desta forma, foi
utilizado o método de Propagação Dupla, que possui como principal objetivo
adicionar novas palavras ao dicionário
padrão do \ac{VADER} \cite{Qiu:2011:OWE:1970420.1970422}.

\section{Método de Propagação Dupla}

O método de Propagação Dupla, proposto por Qiu
\cite{Qiu:2011:OWE:1970420.1970422}, apresenta como objetivo resolver dois
problemas do \ac{NLP}, que são a expansão do dicionário e também a extração dos
alvos de um comentário. Os \textit{``Opinion targets"} ou alvos de opinião, são
palavras as quais sentimentos se referem. Por exemplo, na frase \textit{``Essa
música é muito boa''}, a palavra \textit{``boa''} demonstra o sentimento do
autor com relação a \textit{``música''}, tornando assim, a palavra
\textit{``música''} um alvo de opinião.

A extração dos alvos é interessante pois em uma mesma frase
podemos expressar opiniões sobre diferentes alvos. Além disso, podemos
ter opiniões diferentes sobre diferentes características de um alvo. Por
exemplo, na frase \textit{``Este celular é muito bom, porém a bateria dele é
péssima''}, tem-se que o produto em si é bom, porém, a opinião sobre a bateria
dele é negativa. De forma, a diminuir esses problemas, novas palavras e alvos
são adicionados ao dicionário padrão do \ac{VADER} partir da execução das quatro
tarefas que serão descritas na próxima seção.

\subsection{Adição de Palavras e Alvos ao Dicionário}

A \textbf{primeira tarefa}, é dividida em duas regras, sendo que a primeira
consiste na extração de alvos a partir de palavras que expressam um sentimento e
que já são conhecidas. Ou seja, são verificadas todas as palavras das quais a palavra que
expressa sentimentos depende. Caso essa palavra seja um substantivo, ela será
extraída e adicionada na lista de alvos. Por exemplo na frase \textit{``We have
a great president''} a palavra \textit{``great''} depende de \textit{``president''}, que é um substantivo. Neste caso, a palavra \textit{``president''} será adicionada na lista de alvos.

\[\textit{We have a } \underbrace{\textit{great president.}}_\text{president
\textrightarrow \text{ great}}\]

A segunda regra da primeira tarefa,
consiste em verificar se uma palavra que expressa um sentimento é dependente de
uma segunda palavra que depende de um substantivo.
Por exemplo, na frase \textit{``Trump is a great president.''} a palavra
\textit{``great''} depende da palavra \textit{``is''} que por sua vez depende da
palavra \textit{``president''} que é um substantivo. Neste caso, a palavra
\textit{``president''} será adicionada a lista de alvos.

\[\textit{Trump} \underbrace{\textit{is a great president.}}_\text{president
\textrightarrow \text{ is} \textrightarrow \text{ great}}\]

A \textbf{segunda tarefa} é a extração de novas palavras que expressam
sentimentos. Para isso, são utilizadas duas regras. Na primeira regra, é
verificada se a frase possui alguma palavra que se encontra na lista de alvos de
sentimentos, em caso positivo, é verificado se essa palavra possui algum dependente que seja um adjetivo. Por exemplo, na frase \textit{``Trump is a witty president.''}, a palavra \textit{``president''} foi
extraída na tarefa anterior, porém, a palavra \textit{``witty''} ainda não se
encontra no dicionário de sentimentos.
Desta forma, a palavra \textit{``witty''} será acrescida ao
dicionário de sentimentos uma vez que essa palavra é um adjetivo que não se
encontra no dicionário de sentimentos.


\[\textit{We have a } \underbrace{\textit{witty president.}}_\text{president
\textrightarrow \text{ witty}}\]

A segunda regra da tarefa dois
consiste em verificar se uma palavra alvo possui um dependente, que por sua vez
possui um adjetivo como dependente. Por exemplo, na frase
\textit{``Trump is a wholesome president.''} a palavra \textit{``wholesome''}, que é um
adjetivo, depende de \textit{``is''} que por sua vez depende de
\textit{``president''}. Neste caso, a palavra \textit{``wholesome''}
será adicionada ao dicionário de sentimentos uma vez que essa é um adjetivo que
ainda não existe no dicionário.

\[\textit{Trump} \underbrace{\textit{is a wholesome president.}}_\text{president
\textrightarrow \text{ is} \textrightarrow \text{ wholesome}}\]

A \textbf{terceira tarefa} consiste na extração de palavras alvo a partir de
palavras alvo que já se encontram na lista de alvos. A primeira regra desta
tarefa verifica se a frase possui alguma palavra na lista de alvos, e verifica se essa possui alguma
conjunção. Em caso positivo, a palavra após a conjunção é adicionada na lista
de alvos. Por exemplo, na frase \textit{``We have a great president and
leader."}, a palavra \textit{``president''} que foi extraída através
da primeira tarefa, possui a conjunção \textit{``and''} que a relaciona com a
palavra \textit{``leader''}, a qual não consta na lista de alvos. Neste
caso, a palavra \textit{``leader''} será adicionada a lista de alvos.


\[\textit{We have a great} \underbrace{\textit{president and
leader.}}_\text{president \textrightarrow \text{ and} \textrightarrow \text{
leader}}\]

A segunda regra da terceira tarefa verifica se dois substantivos
possuem uma palavra dependente em comum. No caso de um desses substantivos ser
uma palavra alvo, o outro também será adicionado na lista. Por exemplo, na frase
\textit{``Trump is a great president.''}, a palavra \textit{``president''} que
foi extraída através da primeira tarefa possui a palavra \textit{``is''} como
dependente, bem como a palavra \textit{``Trump''} também possui a palavra
\textit{``is''} como dependente. Desta forma, a palavra \textit{``Trump''} será adicionada a
lista de alvos.

\[\underbrace{\textit{Trump is a great president.}}_\text{Trump
\textrightarrow \text{ is} \textleftarrow \text{ president}}\]



A \textbf{quarta tarefa} tem como objetivo a extração de novos sentimentos a
partir de adjetivos já extraídos. A primeira regra desta tarefa, efetua a extração através
das conjunções presentes no texto. Por exemplo, para a frase \textit{``Trump is
witty and clever.''}, a palavra \textit{``witty''} possui uma conjunção
(\textit{``and''}), que a relaciona com a palavra \textit{``clever''}. Neste
caso, a palavra \textit{``clever''} será adicionada ao dicionário.

\[\textit{Trump} \underbrace{\textit{is witty and clever.}}_\text{witty
\textleftarrow \text{ and} \textrightarrow \text{ clever}}\]


A segunda regra da quarta tarefa, consiste em verificar-se dois adjetivos
dependem de uma mesma palavra. Se uma destas palavras pertence ao dicionário de
sentimentos, a outra também será adicionada ao dicionário. Por exemplo, na frase
\textit{``Trump is witty, clever''}, ambas as palavras \textit{``witty''} e
\textit{``clever''} dependem de \textit{``Trump''}. Neste caso, será extraída a
palavra \textit{``clever''} e adicionada ao dicionário.

\[\textit{Trump} \underbrace{\textit{is witty, clever.}}_\text{witty
\textleftarrow \text{ Trump} \textrightarrow \text{ clever}}\]

Por fim, é verificado se a lista de alvo ou o dicionário de sentimento sofreram
alterações, ou seja, se foram adicionadas novas palavras a essas listas. Em caso
positivo, as quatro tarefas serão executadas até que nenhuma palavra seja
adicionada nas listas.

\subsection{Atribuição de Pontuação às Novas Palavras}

Após a execução das quatro tarefas, se faz necessário definir a
pontuação relativa aos novos sentimentos e aos alvos que foram identificados
através do Método de Propagação Dupla. Essa atribuição é realizada da seguinte forma,
para as palavras que foram extraídas através da quarta tarefa, é utilizada a
mesma pontuação da palavra relacionada a essa nova palavra. Por exemplo, para a frase \textit{``Trump is
witty and clever.''}, a palavra \textit{``witty''} possui uma conjunção
\textit{``and''}, que a relaciona com a palavra \textit{``clever''}. Neste
caso, a palavra \textit{``clever''} terá a mesma pontuação da palavra
\textit{``witty''}.

Para a pontuação das palavras extraídas na tarefa dois, é realizado o seguinte
processo, na extração do alvo é atribuído a este alvo a mesma pontuação do
sentimento relacionado com ele. Por exemplo, na frase \textit{``We have a great
president''}, a palavra \textit{``president''} será adicionada a lista de alvos
de sentimentos com a mesma pontuação de \textit{``great''}, a qual está
relacionada. Após, essa pontuação é atribuída a palavra extraída a partir
da tarefa dois.
Por exemplo, no comentário \textit{``Trump is a wholesome president.''}, na
segunda tarefa, a palavra \textit{``wholesome''}, que foi extraída a partir da
palavra \textit{``president''}, irá receber a mesma pontuação da palavra
\textit{``great''}.

\subsection{Textos utilizados para o método de Propagação Dupla}
\label{sec:textos}

Para utilização do método de Propagação Dupla, é necessário utilizar-se de um
conjunto de textos para que sejam executadas as quatro tarefas. Segundo Qiu
\cite{Qiu:2011:OWE:1970420.1970422}, como as palavras podem ter diferentes
significados em diferentes contextos, é recomendado que este conjunto de textos pertença ao mesmo contexto que está sendo efetuada a análise de sentimentos. Por
exemplo, ao utilizar um conjunto de dados sobre notícias gerais, pode-se
determinar que a palavra \textit{``gucci''} é uma gíria similar a palavra
\textit{``good''}. Porém, o uso dessa palavra em comentários de roupas
apresenta um outro contexto, se referindo a uma marca de roupas.

A fim de se determinar o melhor conjunto de textos para utilização do método de
Propagação Dupla, foram utilizados dois conjuntos de dados de testes. No
primeiro conjunto de textos, considerou-se os comentários de todos os usuários
que comentaram o tópico \textit{``Canada will welcome you,” Trudeau invites
refugees as Trump bans them}. Mais especificamente, foram considerados os
últimos 1000 comentários destes usuários, totalizando 780063 comentários.

No segundo
conjunto de textos, foram considerados os comentários de usuários em tópicos
que apresentam o mesmo tema do tópico \textit{``Canada will welcome you,” Trudeau invites
refugees as Trump bans them}. Mais especificamente, foram utilizados os
seguintes tópicos:

\begin{itemize}
  \item
  \textit{Donald Trump to strip all funding from State Dept team promoting
  women's rights around the world - Leaked plan comes as First Daughter Ivanka
  defends her father's record with women}: esse tópico contém 9246
  comentários e encontra-se disponível em
  \textit{\url{https://www.reddit.com/r/worldnews/comments/67ivae/donald_trump_to_strip_all_funding_from_state_dept/}}.
  Esse tópico refere-se a decisão do presidente dos Estados Unidos da América,
  Donald Trump, em remover os fundos de promoção ao direito das mulheres.
  \item
  \textit{Sweden asks the U.S. to explain Trump comment on
  Sweden}: esse tópico contém 10927
  comentários e encontra-se disponível em
  \textit{\url{https://www.reddit.com/r/worldnews/comments/5uzetf/sweden_asks_the_us_to_explain_trump_comment_on/}}
  e se refere aos comentários feitos do presidente dos Estados Unidos da
  América, Donald Trump, sobre a Suécia.
  
  \item\textit{“Canada will welcome you,” Trudeau invites refugees as Trump bans
  them}: esse tópico contém 9113
  comentários e encontra-se disponível em
  \textit{\url{https://www.reddit.com/r/worldnews/comments/5qqa51/canada_will_welcome_you_trudeau_invites_refugees/}}
  e refere-se a declaração do primeiro ministro canadense sobre decisão de
  receber refugiados. Neste declaração, o primeiro ministro canadense afirma que
  os refugiados serão bem-vindos no Canadá.
\end{itemize}
 
Após a utilização do Método de Propagação Dupla utilizando esses dois conjuntos
de textos, foram obtidos os seguintes resultados para o tópico \textit{``Canada
will welcome you,” Trudeau invites refugees as Trump bans them}:
\begin{itemize}
  \item 58\% de assertividade na utilização do dicionário padrão.
  \item 59\% de assertividade na utilização do dicionário padrão com palavras
  extraídas a partir dos últimos 1000 comentários de todos os usuários que
  comentaram o tópico em questão.
  \item 62\% de assertividade na utilização do dicionário padrão com palavras
  extraídas a partir de comentários postados em tópicos relacionados ao mesmo
  tema.
\end{itemize}

Observa-se que os melhores resultados foram obtidos utilizando-se o método de
Propagação Dupla utilizando textos com assuntos relacionados. Desta forma, essa
será a abordagem utilizada no restante do trabalho.

\section{Restrição de Palavras Alvos}

Verificou-se que após a utilização do Método de Propagação Dupla, que os
comentários que não foram identificados de forma correta são comentários como: \textit{"...These laws are not "racist", morons keep hysterically throwing that word around and
its losing all meaning..."}, aonde que o sentimento expresso é negativo, porém,
a pessoa expressa um sentimento a favor da notícia em questão. 

Para amenizar esse problema, pode-se utilizar uma técnica que consiste em
restringir a análise à determinadas palavras alvos, ou seja, extrair somente os
sentimentos relacionados a uma determinada palavra alvo.
Por exemplo, no tópico \textit{````Canada will welcome you,'' Trudeau invites refugees as Trump bans them''}, é possível que a mesma pessoa expresse uma opinião positiva com relação
ao \textit{Canadá} e negativa com relação a \textit{Trudeau}. Ao se determinar
que desejamos analisar comentários sobre a palavra \textit{``Trudeau''},
pode-se definir \textit{Trudeau} como a palavra alvo para a análise,
sendo consideradas somente as frases relacionadas com essa palavra. Por exemplo, \textit{``Come enjoy our 15\%
tax :D edit: Trudeau is the most fake person ever. He always pulls this shit but
undercuts us Canadians all the time. He ain't getting voted in next
election.''}, será analisada somente a frase dependente da palavra
\textit{``Trudeau''}, que será \textit{``\ldots is the most fake person ever''},
evitando que a primeira frase \textit{``Come enjoy our 15\% tax :D\ldots''}, que
representa um sentimento positivo, afete a análise.

Restringindo a análise a palavra alvo \textit{``Trump''} para extração dos
comentários do tópico \textit{````Canada will welcome you,'' Trudeau invites refugees as Trump
bans them''}, foi obtida a assertividade de 65\%. A partir destes resultados,
optou-se pela utilização do método de \ac{VADER}, combinado com o método de Propagação Dupla e a
restrição de alvos para a execução dos testes que serão apresentados no Capítulo
\ref{cap:impl}.




% \chapter{Implementação do Método \ac{VADER}}
\label{cap:impl}
Como forma de avaliar a implementação realizada para uma posterior avaliação dos
resultados, foi selecionado o tópico: ``Canada will welcome you, Trudeau
invites refugees as Trump bans them''. A partir desse tópico, foram extraídos os
comentários armazenados anteriormente através do Capítulo \ref{cap:Extracao} e
estes foram processados através da ferramenta \ac{NLTK}. A utilização do
\ac{VADER} através do \ac{NLTK} disponibiliza como resultado um objeto contendo
a pontuação positiva //TODO(colocar exemplo), neutra, negativa e composta. 

A partir da
pontuação composta, foram avaliados de forma manual os comentários com o
objetivo de se determinar se a classificação estava sendo feita de forma
correta, obtendo-se 58\% de assertividade.

Parte dos comentários que foram identificados de forma
incorreta apresentam seu sentimento de forma irônica \\TODO EXEMPLO. A
identificação de irônia através da leitura textual se faz difícil mesmo
para um ser-humano visto que muitas vezes este texto pode não apresentar sinais
de humor. Ferramentas que fazem uso de dicionários de palavras
utilizam-se do princípio que o sentimento expresso é o qual é
apresentado na maior parte do texto análisado. No caso de comentários
irônicos como \textit{``Good luck with that.''} a metologia de análise de
sentimentos se apresenta falha, sem possilibidade de classificação correta sem
se apoiar em outros métodos. Também, que a ironia para a análise de sentimentos
é considerado um tópico problemático sendo alvo de diversos estudos
\cite{DBLP:conf/lrec/StranisciBFP16}.

Outros comentários que foram identificados de forma incorreta determinou-se que
o sentimento expresso não se encontrava em dicionário de sentimentos padrão do
\ac{VADER}. Como solução para este problema, foi utilizado o método de
Propagação Dupla para adicionar novas palavras ao dicionário
\cite{Qiu:2011:OWE:1970420.1970422}.

\section{Expansão do Dicionário através do Método de Propagação Dupla}

O método de Propagação Dupla, proposto por Qiu
\cite{Qiu:2011:OWE:1970420.1970422} tem como objetivo solução de dois problemas
do \ac{NLP}, a expansão de dicionário e também a extração dos alvos destas
opiniões. Para este problema é proposta uma solução na qual novas palavras e
alvos são adicionados ao dicionário de forma recursiva a partir de palavras já
conhecidas (dicionário padrão do \ac{VADER}) e também alvos já conhecidos
(palavras relacionadas com as que constam no dicionário do \ac{VADER}).

Essa implementação é feita a partir de quatro sub-tarefas, as quais contém duas
regras cada uma destas. A primeira sub-tarefa é a extração de alvos a partir de
palavras que expressam sentimento já conhecidas. Na primeira regra dessa
dessa sub-tarefa são verificadas as palavras na qual a palavra que
expressa sentimento depende e caso essa seja um substantivo, ela será extraída. Como por exemplo na frase \textit{``We have
a great president''} a palavra \textit{``great''} depende de
\textit{``president''} e essa é um substantivo.

\[\textit{We have a } \underbrace{\textit{great president.}}_\text{president
\textrightarrow \text{ great}}\]

Neste caso, a palavra \textit{``president''} será adicionada a lista de
palavras que são alvos de sentimentos. A segunda regra da primeira sub-tarefa
verifica se uma palavra que expressa sentimentos depende de uma determinada
palavra, que depende de um substantivo. Como por exemplo na frase
\textit{``Trump is a great president.''} a palavra \textit{``great''} depende de
\textit{``is''} que por sua vez depende de \textit{``president''} e essa é um
substantivo.

\[\textit{Trump} \underbrace{\textit{is a great president.}}_\text{president
\textrightarrow \text{ is} \textrightarrow \text{ great}}\]

A segunda sub-tarefa é a extração de novas palavras que expressam sentimentos a
partir dos alvos extraídos anteriormente. Para isso, são utilizadas duas
regras. Na primeira regra dessa sub-tarefa, é verificada se a frase possui uma
palavra existente na lista de alvos de sentimentos, e a partir dessa palavra, é verificado ela possui algum
dependente que seja um adjetivo.


\[\textit{We have a } \underbrace{\textit{witty president.}}_\text{president
\textrightarrow \text{ witty}}\]

A palavra \textit{``president''} foi extraída na sub-tarefa
anterior, porém, a palavra \textit{``witty''} não existe no dicionário de
sentimentos. Ao processar a segunda sub-tarefa, a palavra \textit{``witty''}
será acrescida ao dicionário de sentimentos pois essa é um adjetivo que não
existe ainda no nosso conjunto de palavras. A segunda regra dessa sub-tarefa
verifica se uma palavra alvo possui um dependente que por sua vez possua um
adjetivo como dependente. Como por exemplo na frase
\textit{``Trump is a witty president.''} a palavra \textit{``witty''}, que é um
adjetivo, depende de \textit{``is''} que por sua vez depende de
\textit{``president''}.

\[\textit{Trump} \underbrace{\textit{is a witty president.}}_\text{president
\textrightarrow \text{ is} \textrightarrow \text{ witty}}\]

A terceira sub-tarefa é a extração de palavras alvo a partir de palavras alvo já
extraídas anteriormente. A sua primeira regra verifica se a frase possui uma
palavra da lista de alvos e verifica se essa possuí alguma conjunção. Caso sim,
essa última palavra é adicionada a lista de alvos.


\[\textit{We have a great} \underbrace{\textit{president and
leader.}}_\text{president \textrightarrow \text{ and} \textrightarrow \text{
leader}}\]

Neste caso, a palavra \textit{``president''} que foi extraída através da primeira
sub-tarefa possui a conjunção \textit{``and''} que a relaciona com a palavra
\textit{``leader''}, a qual não consta na lista de palavras alvo e portanto será
adicionada. A segunda regra dessa sub-tarefa verifica se dois substantivos
possuem uma palavra dependente em comum, e caso um desses substantivos seja
uma palavra alvo, o outro também é adicionado na lista.

\[\underbrace{\textit{Trump is a great president.}}_\text{Trump
\textrightarrow \text{ is} \textleftarrow \text{ president}}\]

A palavra \textit{``president''} que foi extraída através da primeira
sub-tarefa possui a palavra \textit{``is''} como dependente, assim como a
palavra \textit{``Trump''}. Neste caso, a palavra \textit{``Trump''} será
adicionada a lista de alvos.

A última sub-tarefa tem como objetivo a extração de opiniões a partir das
opiniões já extraídas. A sua primeira regra, assim como a primeira regra da
terceira sub-tarefa, efetua a extração através das conjunções presentes no
texto.

\[\textit{Trump} \underbrace{\textit{is witty and clever.}}_\text{witty
\textleftarrow \text{ and} \textrightarrow \text{ clever}}\]

Neste caso aonde a palavra \textit{``witty''} foi extraída anteriormente, é
verificada se essa possuí uma conjunção, no caso, a palavra \textit{``and''}, e
extraída a palavra relacionada com essa conjunção. Extraindo a palavra
\textit{``clever''} para o dicionário de sentimentos.

A última regra verifica se dois adjetivos possuem dependência com a
mesma palavra, e caso um desses pertence ao dicionário de sentimentos, o outro
também é adicionado na lista.

\[\textit{Trump} \underbrace{\textit{is witty, clever.}}_\text{witty
\textleftarrow \text{ Trump} \textrightarrow \text{ clever}}\]

Por fim, é verificado se o número de palavras da lista de alvos ou sentimentos
aumentou desde a última execução das quatro sub-tarefas. Caso sim, o conjunto de
sub-tarefas será executado novamente até que nenhuma palavra seja acrescentada a
uma das listas.

A pontuação das novas palavras do dicionário de sentimentos é realizada da
seguinte forma. Para palavras extraídas através da quarta sub-tarefa, é
utilizada a mesma pontuação da palavra relacionada a essa nova palavra. Para
palavras extraídas através da segunda sub-tarefa é utilizada a mesma pontuação
da palavra utilizada para a extração do alvo a qual essa palavra utiliza. Como
por exemplo, na primeira regra foi extraída a palavra \textit{``president''} a
partir de \textit{``great''}. Na segunda sub-tarefa, ao extrair a palavra
\textit{``witty''} através de \textit{``president''}, ela irá receber a mesma
pontução de \textit{``great''}.




% \chapter{Conclusão Final}
\label{cap:conclusao}
Através deste trabalho, observou-se que os métodos mais utilizados para análise
de sentimentos são o Método de Naive Bayes (estatístico) e o Método de \ac{VADER} (simbólico). Dentre
estes, optou-se pela utilização do método de \ac{VADER}, uma vez que de acordo
com a literatura, esse apresenta um desempenho superior ao método de Naive
Bayes, na análise de sentimentos nas avaliações de
produtos da Amazon, editoriais do New York Times e mais importante, na análise
de \textit{Tweets} da rede social Twitter \cite{SentimentinSocialMedia}.

A partir dos resultados obtidos verifica-se que o método de \ac{VADER} sozinho
apresenta resultados insatisfatórios. Desta forma, foi
utilizado o método de Propagação Dupla e também a escolha de palavras alvo para
a análise. 

Foram analisados 6 tópicos divididos em duas categorias,
comentários políticos e comentários de filmes. Estes tópicos foram analisados em
sua assertividade e assertividade por quantidade de caracteres. Os resultados
desta análise nos demonstrou que os comentários sobre filmes, apresentaram maior
assertividade que comentários políticos. Isso se deve ao fato de que nos tópicos relacionados com filmes, é pedido a opinião
ou avaliação dos usuários sobre aquele filme, aumentando a utilização de
expressões simples como \textit{``\ldots este filme foi ruim\ldots''}. Também, a
partir da análise de assertividade por quantidade de caracter, foi possível
demonstrar que o \ac{VADER} tende a perder assertividade quanto maior a
quantidade de caracteres em um comentário. 

De fato, quanto maior o número de
caracteres, menor a assertividade do método. Essas tendências já haviam sido
observadas nos trabalhos de Hutto e Gilbert \cite{conf/icwsm/HuttoG14}.


\section{Trabalhos Futuros}

Como sugestão de trabalhos futuros, a análise de comentários do Reddit através do
\textit{``Naive Bayes''} para que seja possível elencarmos pontos positivos e
negativos deste método em relação ao método de \ac{VADER} na sua utilização
no Reddit.

Sugere-se ainda, a incorporação do
histórico do usuário na análise de sentimentos. Ou seja, para definirmos se um
usuário está sendo irônico, por exemplo, seria interessante verificar se este
sempre elogiou determinado produto, e somente neste determinado comentário ele está criticando este.

% \include{capitulos/capitulo7}

\bibliography{tcc}
\bibliographystyle{abnt}

\end{document}