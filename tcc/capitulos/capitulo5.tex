\chapter{Implementação do Método \ac{VADER}}
\label{cap:impl}
Como forma de avaliar a implementação realizada para uma posterior avaliação dos
resultados, foi selecionado o tópico: ``Canada will welcome you, Trudeau
invites refugees as Trump bans them''. A partir desse tópico, foram extraídos os
comentários armazenados anteriormente através do Capítulo \ref{cap:Extracao} e
estes foram processados através da ferramenta \ac{NLTK}. A utilização do
\ac{VADER} através do \ac{NLTK} disponibiliza como resultado um objeto contendo
a pontuação positiva //TODO(colocar exemplo), neutra, negativa e composta. 

A partir da
pontuação composta, foram avaliados de forma manual os comentários com o
objetivo de se determinar se a classificação estava sendo feita de forma
correta, obtendo-se 58\% de assertividade.

Parte dos comentários que foram identificados de forma
incorreta apresentam seu sentimento de forma irônica \\TODO EXEMPLO. A
identificação de irônia através da leitura textual se faz difícil mesmo
para um ser-humano visto que muitas vezes este texto pode não apresentar sinais
de humor. Ferramentas que fazem uso de dicionários de palavras
utilizam-se do princípio que o sentimento expresso é o qual é
apresentado na maior parte do texto análisado. No caso de comentários
irônicos como \textit{``Good luck with that.''} a metologia de análise de
sentimentos se apresenta falha, sem possilibidade de classificação correta sem
se apoiar em outros métodos. Também, que a ironia para a análise de sentimentos
é considerado um tópico problemático sendo alvo de diversos estudos
\cite{DBLP:conf/lrec/StranisciBFP16}.

Outros comentários que foram identificados de forma incorreta determinou-se que
o sentimento expresso não se encontrava em dicionário de sentimentos padrão do
\ac{VADER}. Como solução para este problema, foi utilizado o método de
Propagação Dupla para adicionar novas palavras ao dicionário
\cite{Qiu:2011:OWE:1970420.1970422}.

\section{Expansão do Dicionário através do Método de Propagação Dupla}

O método de Propagação Dupla, proposto por Qiu
\cite{Qiu:2011:OWE:1970420.1970422} tem como objetivo solução de dois problemas
do \ac{NLP}, a expansão de dicionário e também a extração dos alvos destas
opiniões. Para este problema é proposta uma solução na qual novas palavras e
alvos são adicionados ao dicionário de forma recursiva a partir de palavras já
conhecidas (dicionário padrão do \ac{VADER}) e também alvos já conhecidos
(palavras relacionadas com as que constam no dicionário do \ac{VADER}).

Essa implementação é feita a partir de quatro sub-tarefas, as quais contém duas
regras cada uma destas. A primeira sub-tarefa é a extração de alvos a partir de
palavras que expressam sentimento já conhecidas. Na primeira regra dessa
dessa sub-tarefa são verificadas as palavras na qual a palavra que
expressa sentimento depende e caso essa seja um substantivo, ela será extraída. Como por exemplo na frase \textit{``We have
a great president''} a palavra \textit{``great''} depende de
\textit{``president''} e essa é um substantivo.

\[\textit{We have a } \underbrace{\textit{great president.}}_\text{president
\textrightarrow \text{ great}}\]

Neste caso, a palavra \textit{``president''} será adicionada a lista de
palavras que são alvos de sentimentos. A segunda regra da primeira sub-tarefa
verifica se uma palavra que expressa sentimentos depende de uma determinada
palavra, que depende de um substantivo. Como por exemplo na frase
\textit{``Trump is a great president.''} a palavra \textit{``great''} depende de
\textit{``is''} que por sua vez depende de \textit{``president''} e essa é um
substantivo.

\[\textit{Trump} \underbrace{\textit{is a great president.}}_\text{president
\textrightarrow \text{ is} \textrightarrow \text{ great}}\]

A segunda sub-tarefa é a extração de novas palavras que expressam sentimentos a
partir dos alvos extraídos anteriormente. Para isso, são utilizadas duas
regras. Na primeira regra dessa sub-tarefa, é verificada se a frase possui uma
palavra existente na lista de alvos de sentimentos, e a partir dessa palavra, é verificado ela possui algum
dependente que seja um adjetivo.


\[\textit{We have a } \underbrace{\textit{witty president.}}_\text{president
\textrightarrow \text{ witty}}\]

A palavra \textit{``president''} foi extraída na sub-tarefa
anterior, porém, a palavra \textit{``witty''} não existe no dicionário de
sentimentos. Ao processar a segunda sub-tarefa, a palavra \textit{``witty''}
será acrescida ao dicionário de sentimentos pois essa é um adjetivo que não
existe ainda no nosso conjunto de palavras. A segunda regra dessa sub-tarefa
verifica se uma palavra alvo possui um dependente que por sua vez possua um
adjetivo como dependente. Como por exemplo na frase
\textit{``Trump is a witty president.''} a palavra \textit{``witty''}, que é um
adjetivo, depende de \textit{``is''} que por sua vez depende de
\textit{``president''}.

\[\textit{Trump} \underbrace{\textit{is a witty president.}}_\text{president
\textrightarrow \text{ is} \textrightarrow \text{ witty}}\]

A terceira sub-tarefa é a extração de palavras alvo a partir de palavras alvo já
extraídas anteriormente. A sua primeira regra verifica se a frase possui uma
palavra da lista de alvos e verifica se essa possuí alguma conjunção. Caso sim,
essa última palavra é adicionada a lista de alvos.


\[\textit{We have a great} \underbrace{\textit{president and
leader.}}_\text{president \textrightarrow \text{ and} \textrightarrow \text{
leader}}\]

Neste caso, a palavra \textit{``president''} que foi extraída através da primeira
sub-tarefa possui a conjunção \textit{``and''} que a relaciona com a palavra
\textit{``leader''}, a qual não consta na lista de palavras alvo e portanto será
adicionada. A segunda regra dessa sub-tarefa verifica se dois substantivos
possuem uma palavra dependente em comum, e caso um desses substantivos seja
uma palavra alvo, o outro também é adicionado na lista.

\[\underbrace{\textit{Trump is a great president.}}_\text{Trump
\textrightarrow \text{ is} \textleftarrow \text{ president}}\]

A palavra \textit{``president''} que foi extraída através da primeira
sub-tarefa possui a palavra \textit{``is''} como dependente, assim como a
palavra \textit{``Trump''}. Neste caso, a palavra \textit{``Trump''} será
adicionada a lista de alvos.

A última sub-tarefa tem como objetivo a extração de opiniões a partir das
opiniões já extraídas. A sua primeira regra, assim como a primeira regra da
terceira sub-tarefa, efetua a extração através das conjunções presentes no
texto.

\[\textit{Trump} \underbrace{\textit{is witty and clever.}}_\text{witty
\textleftarrow \text{ and} \textrightarrow \text{ clever}}\]

Neste caso aonde a palavra \textit{``witty''} foi extraída anteriormente, é
verificada se essa possuí uma conjunção, no caso, a palavra \textit{``and''}, e
extraída a palavra relacionada com essa conjunção. Extraindo a palavra
\textit{``clever''} para o dicionário de sentimentos.

A última regra verifica se dois adjetivos possuem dependência com a
mesma palavra, e caso um desses pertence ao dicionário de sentimentos, o outro
também é adicionado na lista.

\[\textit{Trump} \underbrace{\textit{is witty, clever.}}_\text{witty
\textleftarrow \text{ Trump} \textrightarrow \text{ clever}}\]

Por fim, é verificado se o número de palavras da lista de alvos ou sentimentos
aumentou desde a última execução das quatro sub-tarefas. Caso sim, o conjunto de
sub-tarefas será executado novamente até que nenhuma palavra seja acrescentada a
uma das listas.

A pontuação das novas palavras do dicionário de sentimentos é realizada da
seguinte forma. Para palavras extraídas através da quarta sub-tarefa, é
utilizada a mesma pontuação da palavra relacionada a essa nova palavra. Para
palavras extraídas através da segunda sub-tarefa é utilizada a mesma pontuação
da palavra utilizada para a extração do alvo a qual essa palavra utiliza. Como
por exemplo, na primeira regra foi extraída a palavra \textit{``president''} a
partir de \textit{``great''}. Na segunda sub-tarefa, ao extrair a palavra
\textit{``witty''} através de \textit{``president''}, ela irá receber a mesma
pontução de \textit{``great''}.



