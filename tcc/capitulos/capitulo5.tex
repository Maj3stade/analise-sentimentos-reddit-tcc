\chapter{Avaliação dos Resultados}
\label{cap:impl}
De forma a avaliar a implementação desenvolvida, foi selecionado o tópico:
\textit{``Canada will welcome you, Trudeau invites refugees as Trump bans
them''.}
A partir desse tópico, foram extraídos os comentários que foram avaliados de
forma manual. Após, os mesmos foram processados através da ferramenta \ac{NLTK},
obtendo uma assertividade de 56\%.

Observa-se que parte dos comentários que foram identificados de forma
incorreta apresentam irônia, como por exemplo: \textit{``lol Good Luck
Canada''} e similares. A identificação de ironia através da leitura textual se
faz difícil até mesmo para um ser-humano visto que muitas vezes este texto pode
não apresentar sinais de humor. As ferramentas que fazem uso de dicionários de
palavras utilizam-se do princípio que o sentimento expresso é o qual é
apresentado na maior parte do texto análisado. No caso de comentários
irônicos como \textit{``Good luck with that.''} e suas variações, que aparecem
trinta vezes no tópico selecionado, a metologia de análise de sentimentos se
apresenta falha. A ironia para a análise de sentimentos é considerado
um tópico problemático sendo alvo de diversos estudos \cite{DBLP:conf/lrec/StranisciBFP16}.

Já para outros comentários que foram identificados de forma incorreta
determinou-se que o sentimento expresso não se encontrava em dicionário de sentimentos padrão do
\ac{VADER}. Como solução para este problema, foi utilizado o método de
Propagação Dupla para adicionar novas palavras ao dicionário
\cite{Qiu:2011:OWE:1970420.1970422}.

\section{Expansão do Dicionário através do Método de Propagação Dupla}

O método de Propagação Dupla, proposto por Qiu
\cite{Qiu:2011:OWE:1970420.1970422}, tem como objetivo resolver dois
problemas do \ac{NLP}, que são a expansão de dicionário e também a extração dos
alvos destas opiniões. \textit{``Opinion targets"} ou alvos de opinião, são
palavras as quais sentimentos se referem. Por exemplo, na frase ``Essa
música é muito boa", a palavra boa demonstra o sentimento do autor com relação a
música. Tornando assim, a palavra música um alvo de opinião. A extração dos
alvos se faz necessária uma vez que, em determinadas frases podemos ter várias
opiniões expressas sobre diversos alvos, e também para entender opiniões
expressas sobre diferentes características de um mesmo produto. Por exemplo, na
frase ``Este celular é muito bom, porém a bateria dele é péssima", se faz
útil entendermos que o produto em si é bom, porém, a opinião geral sobre a
bateria dele é negativa. Para este problema, é proposta uma solução na qual
novas palavras e alvos são adicionados ao dicionário a partir da execução de quatro tarefas sobre um determinado conjunto de textos.

Cada tarefa contém duas regras, a primeira tarefa, é a extração de alvos a
partir de palavras que expressam sentimento e que já são conhecidas. Ou seja, são verificadas as palavras, na qual a palavra que expressa sentimento depende. Caso
essa palavra seja um substantivo, ela será extraída e adicionada na lista de
alvos de sentimentos. Por exemplo na frase \textit{``We have a great
president''} a palavra \textit{``great''} depende de \textit{``president''}, que
é um substantivo. Neste caso, a palavra \textit{``president''} será adicionada a
lista de palavras que são alvos de sentimentos.

\[\textit{We have a } \underbrace{\textit{great president.}}_\text{president
\textrightarrow \text{ great}}\]

A segunda regra da primeira tarefa
verifica se uma palavra que expressa sentimentos depende de uma determinada
palavra, que depende de um substantivo. Por exemplo, na frase
\textit{``Trump is a great president.''} a palavra \textit{``great''} depende de
\textit{``is''} que por sua vez depende de \textit{``president''} e essa é um
substantivo. Neste caso, a palavra \textit{``president''} será
adicionada a lista de palavras que são alvos de sentimentos.

\[\textit{Trump} \underbrace{\textit{is a great president.}}_\text{president
\textrightarrow \text{ is} \textrightarrow \text{ great}}\]

A segunda tarefa é a extração de novas palavras que expressam sentimentos. Para isso, são utilizadas duas
regras, onde na primeira regra, é verificada se a frase possui
uma palavra existente na lista de alvos de sentimentos, após, é verificado se
essa palavra possui algum dependente que seja um adjetivo. Por exemplo, na frase
\textit{``Trump is a witty president.''}, a palavra \textit{``president''} foi
extraída na tarefa anterior, porém, a palavra \textit{``witty''} não existe no dicionário de
sentimentos. Ao processar a segunda tarefa, a palavra \textit{``witty''}
será acrescida ao dicionário de sentimentos uma vez que essa é um adjetivo que
não existe ainda no dicionário.


\[\textit{We have a } \underbrace{\textit{witty president.}}_\text{president
\textrightarrow \text{ witty}}\]

A segunda regra da tarefa dois
verifica se uma palavra alvo possui um dependente, que por sua vez possua um
adjetivo como dependente. Por exemplo, na frase
\textit{``Trump is a witty president.''} a palavra \textit{``witty''}, que é um
adjetivo, depende de \textit{``is''} que por sua vez depende de
\textit{``president''}. Neste caso, a palavra \textit{``witty''}
será acrescida ao dicionário de sentimentos uma vez que essa é um adjetivo que
não existe ainda no dicionário.

\[\textit{Trump} \underbrace{\textit{is a witty president.}}_\text{president
\textrightarrow \text{ is} \textrightarrow \text{ witty}}\]

A terceira tarefa é a extração de palavras alvo a partir de palavras alvo que
foram extraídas anteriormente. A primeira regra desta tarefa verifica se a frase
possui alguma palavra da lista de alvos e verifica se essa possui alguma
conjunção, em caso positivo, a última palavra é adicionada na lista
de alvos. Por exemplo, na frase \textit{``We have a great president and
leader."}, a palavra \textit{``president''} que foi extraída através
da primeira tarefa, possui a conjunção \textit{``and''} que a relaciona com a
palavra \textit{``leader''}, a qual não consta na lista de palavras alvo. Neste
caso, a palavra \textit{``leader''} será adicionada a lista de palavras alvo.


\[\textit{We have a great} \underbrace{\textit{president and
leader.}}_\text{president \textrightarrow \text{ and} \textrightarrow \text{
leader}}\]

A segunda regra da terceira sub-tarefa verifica se dois substantivos
possuem uma palavra dependente em comum, e caso um desses substantivos seja
uma palavra alvo, o outro também é adicionado na lista. Por exemplo, na frase
\textit{``Trump is a great president.''}, a palavra \textit{``president''} que
foi extraída através da primeira sub-tarefa possui a palavra \textit{``is''} como dependente, assim como a
palavra \textit{``Trump''}. Desta forma, a palavra \textit{``Trump''} será
adicionada a lista de alvos.

\[\underbrace{\textit{Trump is a great president.}}_\text{Trump
\textrightarrow \text{ is} \textleftarrow \text{ president}}\]



A última tarefa tem como objetivo a extração de opiniões a partir de
opiniões que já foram extraídas. A sua primeira regra efetua a extração através
das conjunções presentes no texto. Por exemplo, para a frase \textit{``Trump is
witty and clever.''}, a palavra \textit{``witty''} possui uma conjunção
(\textit{``and''}), relacionando a palavra \textit{``clever''}. Neste caso, será
extraída a palavra \textit{``clever''} e adicionada no dicionário.

\[\textit{Trump} \underbrace{\textit{is witty and clever.}}_\text{witty
\textleftarrow \text{ and} \textrightarrow \text{ clever}}\]


Na última regra da tarefa quatro, é verificado-se dois adjetivos
dependem de uma mesma palavra. Se uma destas palavras pertence ao dicionário de
sentimentos, a outra também será adicionada na lista. Por exemplo, na frase
\textit{``Trump is witty, clever''}, ambas as palavras \textit{``witty''} e
\textit{``clever''} dependem de \textit{``Trump''}. Neste caso, será extraída a
palavra \textit{``clever''} e adicionada ao dicionário.

\[\textit{Trump} \underbrace{\textit{is witty, clever.}}_\text{witty
\textleftarrow \text{ Trump} \textrightarrow \text{ clever}}\]

Por fim, é verificado se o número de palavras da lista de alvos ou sentimentos
aumentou. Em caso positivo, as quatro tarefas serão executadas até que nenhuma palavra seja acrescentada a
uma das listas.

A pontuação das novas palavras do dicionário de sentimentos é realizada
considerando-se: Para as palavras extraídas através da quarta tarefa, é
utilizada a mesma pontuação da palavra relacionada a essa nova palavra. Para palavras extraídas através da segunda tarefa é utilizada a mesma pontuação
da palavra utilizada para a extração do alvo a qual essa palavra utiliza. Por
exemplo, na primeira regra foi extraída a palavra \textit{``president''} a partir de \textit{``great''}. Na segunda tarefa, a palavra
\textit{``witty''}, que foi extraída através de \textit{``president''}, irá
receber a mesma pontuação da palavra \textit{``great''}.

Como visto, para utilização do método de Propagação Dupla, é necessário um
conjunto de textos para que sejam executadas as quatro tarefas. Segundo Qiu
\cite{Qiu:2011:OWE:1970420.1970422}, como as palavras tem diferentes
significados em diferentes contextos, é recomendado que este conjunto de textos
pertença ao mesmo contexto que está sendo efetuada a análise de sentimentos.
A fim de se determinar o melhor conjunto de textos para utilização do método,
foram utilizados dois conjuntos de dados diferentes. No primeiro conjunto de
textos, consideramos como contexto outros comentários do mesmo usuário que
 está sendo efetuada a análise de sentimentos. Para o segundo conjunto de
 textos, foram considerados comentários de usuários em outros tópicos seguindo o
 mesmo tema. Neste avaliação, foram utilizados os comentários presentes nos
 tópicos de política internacional descritos através do Capítulo 4.3.
 

 
Após a utilização do Método de Propagação Dupla utilizando os dois conjuntos de
textos e a avaliação dos comentários foram
obtidos os seguintes resultados:
\begin{itemize}
  \item 58\% de assertividade na utilização do dicionário padrão.
  \item 59\% de assertividade na utilização do dicionário padrão com palavras
  extraídas a partir de comentários do mesmo autor.
  \item 62\% de assertividade na utilização do dicionário padrão com palavras
  extraídas a partir de comentários de um mesmo tema.
\end{itemize}
 
Destaca-se que mesmo com o aumento na assertividade com a utilização de
comentários de tópicos do mesmo domínio, os comentários que não foram
identificados de forma correta são comentários como: \textit{"...These
laws are not "racist", morons keep hysterically throwing that word around and
its losing all meaning..."}, aonde que o sentimento expresso é negativo, porém,
a pessoa expressa um sentimento a favor da notícia em questão. Este problema
pode ser resolvido através da análise não só dos comentários mais também dos
alvos dos sentimentos expressos através destes comentários.
 
\section{Extração de Alvos}

