\chapter{Implementação do Método \ac{VADER}}
\label{cap:impl}
Como forma de avaliar a implementação realizada para uma posterior avaliação dos
resultados, foi selecionado o tópico: ``Canada will welcome you, Trudeau
invites refugees as Trump bans them''. A partir desse tópico, foram extraídos os
comentários armazenados anteriormente através do Capítulo \ref{cap:Extracao} e
estes foram processados através da ferramenta \ac{NLTK}. A utilização do
\ac{VADER} através do \ac{NLTK} disponibiliza como resultado um objeto contendo
a pontuação positiva //TODO(colocar exemplo), neutra, negativa e composta. 

A partir da
pontuação composta, foram avaliados de forma manual os comentários com o
objetivo de se determinar se a classificação estava sendo feita de forma
correta, obtendo-se 58\% de assertividade.

Parte dos comentários que foram identificados de forma
incorreta apresentam seu sentimento de forma irônica \\TODO EXEMPLO. A
identificação de irônia através da leitura textual se faz difícil mesmo
para um ser-humano visto que muitas vezes este texto pode não apresentar sinais
de humor. Ferramentas que fazem uso de dicionários de palavras
utilizam-se do princípio que o sentimento expresso é o qual é
apresentado na maior parte do texto análisado. No caso de comentários
irônicos como \textit{``Good luck with that.''} a metologia de análise de
sentimentos se apresenta falha, sem possilibidade de classificação correta sem
se apoiar em outros métodos. Também, que a ironia para a análise de sentimentos
é considerado um tópico problemático sendo alvo de diversos estudos
\cite{DBLP:conf/lrec/StranisciBFP16}.

Outros comentários que foram identificados de forma incorreta determinou-se que
o sentimento expresso não se encontrava em dicionário de sentimentos padrão do
\ac{VADER}. Como solução para este problema, foi utilizado o método de
Propagação Dupla para adicionar novas palavras ao dicionário
\cite{Qiu:2011:OWE:1970420.1970422}.

\section{Expansão do Dicionário através do Método de Propagação Dupla}

O método de Propagação Dupla, proposto por Qiu
\cite{Qiu:2011:OWE:1970420.1970422} tem como objetivo solução de dois problemas
do \ac{NLP}, a expansão de dicionário e também a extração dos alvos destas
opiniões. Para este problema é proposta uma solução na qual novas palavras e
alvos são adicionados ao dicionário de forma recursiva a partir de palavras já
conhecidas (dicionário padrão do \ac{VADER}) e também alvos já conhecidos
(palavras relacionadas com as que constam no dicionário do \ac{VADER}).
