\chapter{Frameworks}
\label{cap:Frameworks}

Para implementação de um \textit{software} que efetue a análise de sentimentos e
auxilie a categorizar extrair as informações de modo a análisar padrões de
comportamento, se destacam os seguintes \textit{frameworks}:

\begin{itemize}
\item \textit{Stanford's Core NLP Suite}.
\item \textit{Natural Language Toolkit}.
\item \textit{Apache OpenNLP}.
\item \textit{Spacy}.
\end{itemize}

Essa seção visa analisar os métodos
disponíveis nestes para a análise de sentimentos a fim de encontrar \textit{frameworks}
que possibilitem a utilização dos métodos apresentados no
Capítulo \ref{cap:Classificadores} e também possibilitam atender outras
necessidades do processamento de linguagem natural.


\section{Natural Language Toolkit}

O \ac{NLTK} é um \textit{Framework} para Python
criado em 2001 na Universidade de Pensilvânia. Ele contém mais de 50 dicionários
e modelos já treinados incluindo:

\begin{itemize}
  \item \textit{Sentiment Polarity Dataset Version 2.0} - Conjunto de dados já
  classificados que contém mais de 1000 filmes avaliados de forma positiva e
  1000 filmes avaliados de forma negativa.
  \item \textit{SentiWordNet} - Provém um dicionário com as palavras extraídas
  do WordNet já classificadas em positividade, negatividade e objetividade.
  \item \textit{VADER Sentiment Lexicon} - Dicionário especificamente ajustado
  para análise de sentimentos expressos em mídias sociais.
\end{itemize}


\subsection{Análise de Sentimentos}

Para a análise de sentimentos, o \ac{NLTK} já possui implementado os três
classificadores citados anteriormente, \textit{Naive Bayes} e \ac{VADER}.

Podemos utilizar o classificador Naive Bayes a partir da classe

\textbf{nltk.classify.naivebayes.NaiveBayesClassifier} através dos seguintes métodos:

\begin{itemize}
  \item \textit{classify(featureset)} - Classifica a partir de um conjunto de
  atributos.
  \item \textit{most\_informative\_features(n=100)} - A partir de um
  classificador treinado, retorna os atributos mais relevantes.
  \item \textit{train(trainingset)} - Treina um classificador a partir de um \textit{training set}.
\end{itemize}

Para utilização do \ac{VADER} é utilizada a classe
\textit{SentimentIntensityAnalyzer} do módulo \textit{vaderSentiment} através
do método \textit{polarity\_scores}. Este método recebe uma frase e retorna um
objeto contendo a intensidade positiva, neutra e negativa da frase.

O \textit{framework}
também contém um pacote contendo classes úteis para a análise de sentimentos
chamado de \textit{nltk.sentiment}. Nesse pacote temos os seguintes módulos:


\begin{itemize}
  \item Classe \textit{nltk.sentiment.sentiment\_analyzer.SentimentAnalyzer} -
  Ferramentas para facilitar e implementar análise de sentimentos,
  especialmente para demonstrações e ensino.
  \item Módulo \textit{nltk.sentiment.util} - Contém diversas classes de
  demonstrações e utilitários como conversão de \textit{json} para \textit{csv}.
\end{itemize}

\section{Stanford CoreNLP}

O Stanford CoreNLP é um conjunto de ferramentas escrito em Java para
processamento de linguagem natural. Dentre essas ferramentas, estão incluídos:
\textit{Part-of-Speech Tagging} ou classificação gramatical, reconhecimento de
entidade e análise de sentimentos. Também possui suporte a diversas linguas além
do inglês, como: árabe, chinês, francês, alemão e espanhol.

\subsection{Análise de Sentimentos}

A análise de sentimentos do Stanford CoreNLP é realizada através de um novo
modelo de rede neural construído em cima de estruturas gramaticais chamado de
\ac{RNTN}. Seu modelo é treinado a partir do \textit{Sentiment Treebank}, um
banco de dados que possui 215.154 orações distribuidas em 11.855 árvores de
frases com sentimentos já classificados.

\begin{figure}[!htbp]
 \centering
 \includegraphics[height=225px]{imagens/corenlp.png}
 \caption{Frase já classificada disponível no Sentiment Treebank}
 \label{fig:corenlp}
\end{figure}

A sua utilização pode ser feita de diversas formas, como linha de comando,
através de um servidor \textit{web} e através de sua API java:
\newpage

\begin{figure}[!htbp]
 \centering
 \includegraphics[height=120px]{imagens/corenlp1.png}
 \caption{Exemplo de implementação}
 \label{fig:corenlp}
\end{figure}

Como resultado, o console java irá imprimir que a frase é muito positiva ou
\textit{Very positive}.

