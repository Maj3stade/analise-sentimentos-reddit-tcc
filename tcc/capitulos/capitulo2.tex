\chapter{Frameworks}
\label{cap:Frameworks}

A implementação de um software de análise de sentimentos pode ser feita através
de um software escrito a partir de seu início, utilizando como base os métodos
já existentes ou também pode ser feito utilizando \textit{frameworks} já
disponíveis. Métodos estatísticos como o Naive Bayes analisado no Capítulo
\ref{cap:Classificadores} podem ser encontrados tanto em implementações de
aprendizado de máquina, quanto implementações para voltadas para o \ac{NLP}.
Já métodos simbólicos, por suas implementações serem específicas para a o
\ac{NLP} ou análise de sentimentos, somente encontramos implementações destes em
\textit{frameworks} de \ac{NLP}.

O método selecionado para a análise de sentimentos, \ac{VADER}, é
disponibilizado como um \textit{package} Python e também através do
\textit{framework} \ac{NLTK}, o qual é considerado um \textit{kit} de
ferramentas não só para análise de sentimentos mas como o \ac{NLP} como um todo.

Por trazer diversas outras implementações que possam a vir facilitar a
identificação de padrões de sentimentos, foi escolhido utilizar o \ac{NLTK} para
acessar o \ac{VADER} ao invés de somente a utilização do próprio
\textit{package} desse.

O \ac{NLTK} é um \textit{framework open source} para Python
criado em 2001 na Universidade de Pensilvânia atualmente utilizado por mais de
30 universidades em diversos
países\footnote{\url{https://docs.google.com/document/d/1eYubSwLkpB7ZgfQVxxAwgsmAqS__BRfbMyP9qV6ngD8/edit}}.
Esse apresenta tanto métodos estatísticos, como \ac{MaxEnt} e Naive Bayes como métodos simbólicos, contendo
mais de 50 dicionários e modelos já treinados como \textit{Sentiment Polarity Dataset Version 2.0}, conjunto de dados já classificados que contém mais de 1000 filmes avaliados de forma positiva e 1000 filmes avaliados de forma negativa, \textit{SentiWordNet},
um dicionário com as palavras extraídas do WordNet já classificadas em
positividade, negatividade e objetividade e \ac{VADER} o qual foi selecionado
para aplicação da análise de sentimentos.

Além da execução de análise de sentimentos, o \textit{framework} executa outras
tarefas pertencentes ao \ac{NLP} como o reconhecimento de entidade mencionada e
análise léxica.


