\chapter{Introdução}
\label{chap:introducao}
A linguagem é a forma com que nós nos comunicamos, seja ela escrita ou
falada. De fato, a linguagem é a forma como
expressamos nossas idéias, sentimentos e experiências. O Processamento de
Linguagem Natural, é o termo utilizado para descrever um software ou componente
de hardware que tem como função analisar a linguagem escrita ou falada
\cite{jacksonmoulinier2007}.

Existem duas abordagens para o Processamento de Linguagem Natural, sendo que a
primeira delas é chamada de simbólica (ou racionalista) e a outra de empírica
(ou estatística).
A primeira abordagem consiste em uma série de regras para a manipulação de
símbolos, como as regras gramaticais, que permitem identificar se uma frase está malformada ou não. A abordagem empírica está centrada na análise estatística da linguagem através de grandes quantidades
de textos, como por exemplo, a utilização de modelos de Markov para reconhecer padrões
na escrita \cite{jacksonmoulinier2007}.

Existem diversos \textit{frameworks open source} que facilitam o desenvolvimento
de \textit{softwares} para o Processamento de Linguagem Natural, sendo que
dentre esses se destacam \textit{Stanford's Core NLP Suite} \cite{corenlp},
\textit{Natural Language Toolkit} \cite{nltk}, \textit{Apache OpenNLP} \cite{opennlp} e \textit{Spacy}
\cite{spacy}.
Esses \textit{frameworks} nos permitem, entre outras coisas, efetuar análise de
sentimentos, identificar tópicos e conteúdos.

A rede social Reddit é o vigésimo terceiro \textit{website} mais acessado na
Internet e o sétimo mais acessado nos Estados Unidos da América \cite{alexa}.
Através deste \textit{website}, seus usuários podem criar ou se inscrever em
comunidades, também conhecidas como \textit{subreddits}.
Uma vez que as comunidades são criadas pelos próprios usuários, podemos encontrar
comunidades sobre todos os assuntos, sejam notícias do mundo, comunidades
partidárias, comunidades criadas para pessoas de uma mesma localidade, comunidades
de imagens engraçadas, etc.

Nestas comunidades é possível visualizar e comentar \textit{links} enviados por
outros usuários.
Além disso, o usuário pode efetuar um voto de forma positiva, caso acredite que
aquele \textit{link} é útil para a comunidade, ou um voto negativo em caso
contrário.
Uma vez que os próprios usuários podem submeter \textit{links}, os eventos e
notícias de todo o mundo são reportados no \textit{website}, como exemplo,
pode-se citar as eleições ocorridas no ano de 2016 nos Estados Unidos e o
tiroteio ocorrido em Paris em 15 de Novembro de 2015.

Dentro deste contexto neste trabalho será desenvolvido um software que permita
realizar a análise dos comentários do \textit{website} Reddit. Mais
especificamente os comentários do Reddit serão analisados com o objetivo de
identificar padrões de sentimentos, ou seja, determinar se a opinião expressada com relação a um determinado tópico é neutra, positiva ou negativa.

\section{Objetivos do Trabalho}

Este trabalho tem como objetivo a análise dos comentários disponíveis no
\textit{website} Reddit, identificando padrões de sentimentos entre os
usuários de suas comunidades. De forma a atingir o objetivo principal desse
trabalho, os seguintes objetivos específicos devem ser realizados:
\begin{itemize}
  \item Desenvolver uma ferramenta para o Processamento Natural de Linguagem através de
\textit{frameworks} já existentes.
 \item Construção de uma base de dados a partir do \textit{website} Reddit.
 \item Efetuar o processamento da base de dados criado utilizando-se a
 ferramenta desenvolvida.
\end{itemize}

\section{Estrutura do Trabalho}
%O presente trabalho está estruturado da seguinte forma: no Capítulo
% \ref{cap:Animacao} será apresentada a história da animação, seu surgimento e conceitos. No Capítulo \ref{cap:Blender} são apresentadas informações sobre o \textit{software} Blender, assim como os métodos disponíveis nesse para a criação de animações 3D. No Capítulo \ref{cap:Kinect} será apresentado o sensor Kinect, assim como seu funcionamento e alguns conceitos sobre as imagens capturadas. No Capítulo \ref{cap:DesenvKinect} serão apresentadas as bibliotecas e ferramentas para o desenvolvimento de aplicações utilizando Kinect. Além disso, é realizado um comparativo, onde são apresentadas as vantagens e desvantagens de cada uma dessas ferramentas. No Capítulo \ref{cap:AplicaIntegra} serão apresentados alguns \textit{softwares} que já fazem a integração entre o sensor e a ferramenta, assim como o funcionamento desses. No Capítulo \ref{cap:Implementacao} serão apresentados os detalhes da solução desenvolvida e da animação gerada, bem como são apresentados os testes realizados e os resultados obtidos. Por fim, no Capítulo \ref{cap:Consideracoes} são apresentadas as considerações finais e sugestões de trabalhos futuros.
