\chapter{Conclusão Final}
\label{cap:conclusao}
Através deste trabalho, observou-se que os métodos mais utilizados para análise
de sentimentos são o Método de Naive Bayes (estatístico) e o Método de \ac{VADER} (simbólico). Dentre
estes, optou-se pela utilização do método de \ac{VADER}, uma vez que de acordo
com a literatura, esse apresenta um desempenho superior ao método de Naive
Bayes, na análise de sentimentos nas avaliações de
produtos da Amazon, editoriais do New York Times e mais importante, na análise
de \textit{Tweets} da rede social Twitter \cite{SentimentinSocialMedia}.

Com a utilização do \ac{VADER} através do \ac{NLTK} para a análise de
comentários do Reddit em um conjunto de dados de validação, pode-se verificar
que somente a utilização do \ac{VADER}, neste contexto, nos traz resultados
abaixo do que foi encontrado na literatura. 

Para a solução deste problema, foi
utilizado o método de Propagação Dupla e também a escolha de palavras alvo para
a análise. Desta forma, foram analisados 6 tópicos divididos em duas categorias,
comentários políticos e comentários de filmes. Estes tópicos foram analisados em
sua assertividade e assertividade por quantidade de caracteres.

Os resultados desta análise nos demonstrou que os comentários sobre filmes,
apresentaram maior assertividade que comentários políticos. Isso se deve ao fato
de que nos tópicos relacionados com filmes, é pedido a opinião
ou avaliação dos usuários sobre aquele filme, aumentando a utilização de
expressões simples como ``\ldots este filme foi ruim\ldots''.

Também, a partir da análise de assertividade por quantidade de caracter, foi
possível demonstrar que o \ac{VADER} tende a perder assertividade
quanto maior a quantidade de caracteres em um comentário.

Por fim, podemos concluir que é possível a criação de uma ferramenta que executa
análise de sentimentos na rede social Reddit, como a criada por este trabalho,
porém, a assertividade dessa ferramenta inclina-se conforme o contexto dos
comentários, e também a quantidade de caracteres deste. Por isso, se
recomenda que a utilização desta ferramenta seja somente em tópicos controlados,
nos quais já se foi provado a assertividade do \ac{VADER} sobre este, como análise de comentários sobre
filmes.

\section{Trabalhos Futuros}

Como sugestão de trabalhos futuros, tem-se a resolução de problemas encontrados
durante a elaboração deste trabalho, como a modificação de sentimentos de palavras do dicionário
\ac{VADER} conforme o contexto e também a possibilidade do \ac{VADER} expressar
resultados não somente através de uma pontuação sobre um corpo de texto, mas sim
sobre os alvos de sentimentos daquele corpo de texto.

Também, se faz interessante a análise de comentários do Reddit através do
\textit{``Naive Bayes''} para que seja possível elencarmos pontos positivos e
negativos deste com relação ao \ac{VADER} na sua utilização no Reddit.

Por fim, tem-se como sugestão de trabalho futuro, a incorporação do histórico do
usuário na análise de sentimentos. Como por exemplo, determinar que o usuário
esta sendo irônico ao verificar que no histórico deste, ele sempre elogiou
determinado produto, e somente neste determinado comentário ele está criticando
este.
