\chapter{Conclusão Final}
\label{cap:conclusao}
A \ac{NLP} tem como objetivo a análise de linguagem natural, seja
essa escrita ou falada. Dentre diversas tarefas que ela executa, tem-se a
análise de sentimentos, a qual recebeu destaque nos últimos anos devido ao fato das pessoas cada vez se comunicarem através de redes sociais,
gerando um grande volume de dados. A análise e quantificação da opinião expressa por esses dados, seja
por fins políticos, comerciais ou quaisquer outros, se torna díficil devido a
essa grande quantidade de dados.

Observou-se que os métodos mais utilizados para análise de sentimentos são o
Método de Naive Bayes (estatístico) e o Método de \ac{VADER} (simbólico). Dentre
estes, optou-se pela utilização do método de \ac{VADER}, uma vez que de acordo
com a literatura, esse apresenta um desempenho superior ao método de Naive
Bayes. De fato, esse mostrou-se superior na análise de sentimentos nas avaliações de
produtos da Amazon, editoriais do New York Times e mais importante, na análise
de \textit{Tweets} da rede social Twitter \cite{SentimentinSocialMedia}. A
justificativa para isso, se dá ao fato de métodos estatísticos necessitarem de um \textit{training set}
especializado para obter resultados similares ou superiores aos métodos
simbólicos. 

Além disso, optou-se pela utilização do método de \ac{VADER}, devido
ao fato deste não necessitar da criação de um \textit{training set} específico
para a análise de sentimentos. A necessidade da criação de um \textit{training
set} específico para cada tema inviabilizaria o desenvolvimento deste trabalho,
uma vez que neste serão analisados 15 tópicos com temas distintos. Para
implementação do método \ac{VADER} optou-se pela biblioteca \ac{NLTK}. A
utilização da biblioteca \ac{NLTK} permitirá a utilização futura de outros
métodos de \ac{NLP}, bem como um estudo da performance do Método de \ac{VADER} e esses outros métodos.
 
Por fim, se fez necessária a criação de uma base de dados para armazenar os
tópicos e os comentários disponibilizados pela rede social Reddit. Para isso,
foi desenvolvido um \textit{crawler} (robô) que é responsável por extrair os
comentários relacionados a um tópico na rede social Reddit e armazenar em um
banco de dados MySQL. Esse robô foi desenvolvido na linguagem Java e utiliza a
API da rede social Reddit para extrair as informações de um tópico. Após, ele
utiliza o \textit{framework} Hibernate para armazenar os dados extraídos na
base de dados MySQL.

Na segunda etapa deste trabalho, será utilizado o \textit{framework} \ac{NLTK}
para efetuar a análise de sentimento sobre a base de dados criada. Essa análise
tem como principal objetivo identificar padrões de sentimentos entre
usuários e comunidades da rede Reddit.

\section{Atividades e Cronograma}

Na Tabela \ref{tab:tcc1} tem-se o cronongrama das atividades realizadas durante
o TCC I. Como pode ser observado, todas as tarefas programadas foram
realizadas.
\begin{enumerate}
\item Estudo de algoritmos para o processamento de texto e também análise de
sentimentos.
\item Análise das ferramentas já existentes.
\item Análise da API do Reddit.
\item Construção de um software para extração dos dados da API.
\item Extração e criação da base de dados.
\item Redação da monografia TCC I.
\item Apresentação TCC I.
\end{enumerate}

\renewcommand{\arraystretch}{2}
\newcolumntype{Y}{>{\centering\arraybackslash}X}
\begin{table}[!htb]
\begin{tabularx}{0.9\textwidth}{Y|Y|Y|Y|Y|Y|Y|Y|Y|Y|Y|}
& \multicolumn{2}{|c|}{Mar} & \multicolumn{2}{|c|}{Abr} &
\multicolumn{2}{|c|}{Mai} & \multicolumn{2}{|c|}{Jun} &
\multicolumn{2}{|c|}{Jul}
\\
\midrule
1 & \cellcolor{black!80} & \cellcolor{black!80} & & & & & & & & \\
2 &  & \cellcolor{black!80} & \cellcolor{black!80} & & & & & & &\\
3 &  &  &  & \cellcolor{black!80} & & & & & &\\
4 &  &  &  &  & \cellcolor{black!80} & & & &  &\\
5 &  &  &  &  &  & \cellcolor{black!80} & \cellcolor{black!80} & & &\\
6 &  & \cellcolor{black!80}  & \cellcolor{black!80}  &  \cellcolor{black!80} & 
\cellcolor{black!80} & \cellcolor{black!80} & \cellcolor{black!80} & \cellcolor{black!80} &  &\\
7 &  &  &  &  &  & & & & \cellcolor{black!80} &\\
\end{tabularx}

\caption{Cronograma do TCC I.}
\label{tab:tcc1}
\end{table}

Já na Tabela \ref{tab:tcc2} tem-se as atividades a serem desenvolvidas no TCC
II:

\begin{enumerate}
\item Implementação do software de Processamento de Linguagem Natural para a
análise de sentimentos na base de dados criada.
\item Análise dos resultados obtidos.
\item Redação da monografia TCC II.
\item Apresentação do TCC II.
\end{enumerate}

%loucura
\makeatletter
\setlength{\@fptop}{0pt}
\makeatother
\newcolumntype{Y}{>{\centering\arraybackslash}X}
\begin{table}[ht!]
\begin{tabularx}{0.9\textwidth}{Y|Y|Y|Y|Y|Y|Y|Y|Y|Y|Y|}
& \multicolumn{2}{|c|}{Ago} & \multicolumn{2}{|c|}{Set} &
\multicolumn{2}{|c|}{Out} & \multicolumn{2}{|c|}{Nov} &
\multicolumn{2}{|c|}{Dez}
\\
\midrule
1 & \cellcolor{black!80} & \cellcolor{black!80} & \cellcolor{black!80} &
\cellcolor{black!80} & & & & & & \\
2 &  & & & \cellcolor{black!80} & \cellcolor{black!80} & \cellcolor{black!80} &
& & &\\
3 &  & \cellcolor{black!80} & \cellcolor{black!80} & \cellcolor{black!80} &
\cellcolor{black!80} & \cellcolor{black!80} & \cellcolor{black!80}
& \cellcolor{black!80} & &\\
4 &  &  &  &  &  & & & & \cellcolor{black!80} &\\
\end{tabularx}

\caption{Cronograma do TCC II.}
\label{tab:tcc2}
\end{table}

% No Capítulo \ref{cap:Processamento} foram introduzidos dois tipos de métodos
% distintos para o \ac{NLP}, métodos simbólicos e
% métodos estatísticos, os quais foram estudados para a análise de sentimentos
% através do Capítulo \ref{cap:Classificadores}.
% 
% Através do Capítulo \ref{cap:Classificadores}, foram comparados um método
% simbólico e outro método estatístico a fim de se determinar qual apresenta melhor performance na análise de sentimentos
% aplicada em uma rede social, sendo que a literatura apontou que o método mais
% assertivo é o método \ac{VADER}, o qual está disponível através do
% \textit{framework} \ac{NLTK}.
% 
% Já no Capítulo \ref{cap:banco}, foi apresentada como funciona a rede social
% Reddit, os tópicos selecionados para a análise de sentimentos, e por fim foi
% apresentado a forma na qual esses tópicos serão extraídos para população da base
% de dados.
% 
% A partir das informações demonstradas através deste, deverá ser possível criar
% um \textit{software} que efetue a análise de dados utilizando o \ac{VADER},
% através do \textit{framework} \ac{NLTK}, aplicada nos tópicos demonstrados no
% Capítulo \ref{cap:banco}.
