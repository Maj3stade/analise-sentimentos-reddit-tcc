\chapter{Conclusão Final}
\label{cap:conclusao}
Através deste trabalho, observou-se que os métodos mais utilizados para análise
de sentimentos são o Método de Naive Bayes (estatístico) e o Método de \ac{VADER} (simbólico). Dentre
estes, optou-se pela utilização do método de \ac{VADER}, uma vez que de acordo
com a literatura, esse apresenta um desempenho superior ao método de Naive
Bayes, na análise de sentimentos nas avaliações de
produtos da Amazon, editoriais do New York Times e mais importante, na análise
de \textit{Tweets} da rede social Twitter \cite{SentimentinSocialMedia}.

A partir dos resultados obtidos verifica-se que a utilização do método de
\ac{VADER} isolado apresenta resultados insatisfatórios. Desta forma, foi
utilizado o método de Propagação Dupla e combinado com a escolha de palavras
alvo para a análise. Neste caso, os resultados apresentaram melhorias de mais
7\%. De fato, no caso do Método de \ac{VADER} isolado, obteve-se uma
assertividade de 58\%. Já utilizando o Método de Propagação Dupla com palavras
alvos, obteve-se uma assertividade de 65\%.

Foram analisados 6 tópicos divididos em duas categorias,
comentários políticos e comentários de filmes. Estes tópicos foram analisados
por completo e posteriormente, considerando o número de caracteres. Os
resultados obtidos demonstraram que a aplicação do método desenvolvido sobre
comentários de filmes apresenta uma maior assertividade do que a aplicação do
método sobre comentários políticos. Isso se deve ao fato de que nos tópicos
relacionados com filmes, é pedido a opinião ou avaliação dos usuários sobre
aquele filme, aumentando a utilização de expressões simples como
\textit{``\ldots este filme foi ruim\ldots''}. Já a partir da análise de
assertividade por quantidade de caracter, foi possível concluir que o \ac{VADER}
tende a perder assertividade com o aumento no número de caracteres em um
comentário. De fato, quanto maior o número de
caracteres, menor a assertividade do método. Essas tendências já haviam sido
observadas nos trabalhos de Hutto e Gilbert \cite{conf/icwsm/HuttoG14}.


\section{Trabalhos Futuros}

Como trabalhos futuros, segere-se a análise de comentários do Reddit através
do \textit{``Naive Bayes''} de forma a comparar com o método de \ac{VADER} na
sua utilização no Reddit.

Sugere-se ainda, a incorporação do
histórico do usuário na análise de sentimentos. Ou seja, para definirmos se um
usuário está sendo irônico, por exemplo, seria interessante verificar se este
sempre elogiou determinado produto, e somente neste determinado comentário ele está criticando este.
